\apendice{Especificación de Requisitos}

\section{Introducción}

En este anexo se recoge la especificación de requisitos de la aplicación web que se ha desarrollado.

Este documento tiene el objetivo doble de servir de \textit{contrato} con el cliente y como base de documentación para el análisis de la aplicación.

\section{Objetivos generales}

El proyecto tiene los siguientes objetivos generales:

\begin{itemize}
	\item Desarrollar una aplicación web que permita la realización de cuestionarios para evaluar \textit{Soft Skills}.
	\item Permitir la carga de usuarios tanto desde una carpeta del sistema como desde la interfaz web.
	\item Almacenar los resultados de los cuestionarios de forma anónima con fines estadísticos para los creadores.
	\item La descarga completa de los resultados en un documento correctamente estructurado y que sea sencillo de modelar.
	\item La aplicación web tiene que tener un diseño \textit{responsive} que extienda su funcionalidad a diferentes dispositivos.
	\item La aplicación tiene que tener ser multi-idioma, para poder llegar a más usuarios.
\end{itemize}

\section{Catálogo de requisitos}

Se van a enumerar los requisitos separándolos en dos categorías: requisitos funcionales y no funcionales.

\subsection{Requisitos funcionales}
\begin{itemize}
	\item \textbf{RF-1 Gestión de cuestionarios}: La aplicación web tiene que ser capaz de gestionar cuestionarios.
	\begin{itemize}
		\item \textbf{RF-1.1 Carga de cuestionario}: Se tienen que poder cargar nuevos cuestionarios en la aplicación web.
		\item \textbf{RF-1.2 Realización de cuestionario}: Los usuarios deben poder realizar los cuestionarios disponibles.
	\end{itemize}
	\item \textbf{RF-2 Gestión de usuarios}: La aplicación web tiene que ser capaz de gestionar usuarios.
		\begin{itemize}
		\item \textbf{RF-2.1 Registro de usuario}: Los usuarios deben poder registrarse en la aplicación.
		\item \textbf{RF-2.2 Modificación de datos}: Los usuarios deben poder editar sus datos.
		\item \textbf{RF-2.3 Eliminación de usuario}: Los usuarios tienen que poder darse de baja en la aplicación.
	\end{itemize}
	\item \textbf{RF-3 Gestión de resultados}: La apicación web tiene que permitir la gestión de los resultados obtenidos.
		\begin{itemize}
		\item \textbf{RF-3.1 Almacenamiento de resultados}: Los resultados se tienen que almacenar de forma persistente en la aplicación.
		\item \textbf{RF-3.2 Descarga de resultados}: Los resultatos se tienen que poder descargar de la aplicación.
	\end{itemize}
	\item \textbf{RF-4 Diseño internacional}: La aplicación web tiene que ser capaz de adaptarse a varios idiomas.
		\begin{itemize}
		\item \textbf{RF-4.1 Selección de idioma}: La aplicación debe permitir la selección del idioma.
		\item \textbf{RF-4.2 Idioma de los cuestionarios}: La aplicación debe permitir la carga de cuestionarios en varios idiomas.
	\end{itemize}
	\item \textbf{RF-5 Diseño responsable}: La aplicación tiene que ser capaz de adaptarse a diferentes dispositivos.
\end{itemize}



\section{Especificación de requisitos}


Una muestra de cómo podría ser una tabla de casos de uso:

% Caso de Uso 1 -> Consultar Experimentos.
\begin{table}[p]
	\centering
	\begin{tabularx}{\linewidth}{ p{0.21\columnwidth} p{0.71\columnwidth} }
		\toprule
		\textbf{CU-1}    & \textbf{Ejemplo de caso de uso}\\
		\toprule
		\textbf{Versión}              & 1.0    \\
		\textbf{Autor}                & Alumno \\
		\textbf{Requisitos asociados} & RF-xx, RF-xx \\
		\textbf{Descripción}          & La descripción del CU \\
		\textbf{Precondición}         & Precondiciones (podría haber más de una) \\
		\textbf{Acciones}             &
		\begin{enumerate}
			\def\labelenumi{\arabic{enumi}.}
			\tightlist
			\item Pasos del CU
			\item Pasos del CU (añadir tantos como sean necesarios)
		\end{enumerate}\\
		\textbf{Postcondición}        & Postcondiciones (podría haber más de una) \\
		\textbf{Excepciones}          & Excepciones \\
		\textbf{Importancia}          & Alta o Media o Baja... \\
		\bottomrule
	\end{tabularx}
	\caption{CU-1 Nombre del caso de uso.}
\end{table}


