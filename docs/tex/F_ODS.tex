\apendice{Anexo de sostenibilización curricular}

\section{Introducción}
Mi Trabajo de Fin de Grado (TFG) se centró en la creación de una aplicación web utilizando Spring Boot y Docker para la gestión y realización de cuestionarios sobre "soft skills". A lo largo de este proyecto, he integrado aspectos de sostenibilidad que han sido fundamentales tanto en el diseño como en la implementación de la aplicación. En este anexo, reflexionaré sobre cómo he adquirido y aplicado competencias de sostenibilidad durante el desarrollo de mi TFG.

Se ha utilizado como base para la sostenibilidad el documento \url{https://www.crue.org/wp-content/uploads/2020/02/Directrices_Sosteniblidad_Crue2012.pdf}.

\section{Comprensión de la Sostenibilidad}
Durante el desarrollo de mi TFG, he ampliado mi comprensión de la sostenibilidad, abarcando no solo el aspecto ambiental, sino también las dimensiones económica y social. Este enfoque integral es esencial para desarrollar tecnologías que no solo sean eficientes, sino también responsables y beneficiosas a largo plazo.

\section{Competencias de Sostenibilidad Adquiridas}
\subsection{Desarrollo de Software Sostenible:}

La sostenibilidad en el desarrollo de software implica crear aplicaciones que sean eficientes en el uso de recursos. En mi TFG, he aprendido a optimizar el rendimiento de la aplicación, minimizando el consumo de recursos del servidor y mejorando la eficiencia energética. Utilizar Docker ha permitido una implementación más eficiente y escalable, reduciendo el uso de infraestructura y recursos.

\subsection{Evaluación de Impacto Ambiental:}
A través del uso de tecnologías como Docker, he aprendido a reducir el impacto ambiental del desarrollo y despliegue de aplicaciones. Docker permite la creación de contenedores ligeros que optimizan el uso de recursos, lo cual es esencial para la sostenibilidad tecnológica. Esta práctica no solo mejora la eficiencia, sino que también disminuye la huella de carbono asociada con el uso de servidores físicos.

\subsection{Inclusión de Principios de Responsabilidad Social:}
Mi TFG se centra en la evaluación de "soft skills", que son cruciales para el desarrollo personal y profesional de los individuos. Promover estas habilidades tiene un impacto social positivo, contribuyendo al bienestar y la productividad de las personas en diversos entornos laborales. Esto refleja una dimensión social de la sostenibilidad, ya que fomenta el desarrollo integral de las personas.

\subsection{Implementación de Prácticas de Código Abierto:}
Utilizar y contribuir a proyectos de código abierto es una práctica sostenible que promueve la colaboración y el intercambio de conocimientos. En mi TFG, he integrado bibliotecas y herramientas de código abierto, lo que no solo ha reducido los costos de desarrollo, sino que también ha fomentado una comunidad de desarrollo más inclusiva y colaborativa.

\section{Aplicación de Competencias al TFG}
En mi TFG, la sostenibilidad se ha manifestado en varios aspectos:

\subsection{Eficiencia del Código:}
He optimizado el código para asegurar que la aplicación funcione de manera eficiente, lo que incluye la gestión de memoria y el uso eficiente de recursos del servidor. Esto no solo mejora el rendimiento, sino que también reduce el consumo energético.

\subsection{Uso de Docker:}
Docker ha sido fundamental para garantizar una implementación sostenible. Al crear contenedores, he podido aislar y gestionar dependencias de manera más eficiente, reduciendo la necesidad de infraestructura adicional y mejorando la escalabilidad de la aplicación.

\subsection{Accesibilidad y Usabilidad:}
He diseñado la aplicación con un enfoque en la accesibilidad y usabilidad, asegurando que sea fácil de usar para todas las personas, independientemente de sus habilidades técnicas. Esto contribuye a la inclusión digital y mejora la equidad en el acceso a tecnologías educativas.

\subsection{Promoción de "Soft Skills":}
La aplicación está diseñada para evaluar y mejorar las "soft skills" de los usuarios, lo que tiene un impacto positivo en su desarrollo personal y profesional. Al promover habilidades como la comunicación, el trabajo en equipo y la empatía, la aplicación contribuye a la creación de entornos de trabajo más sostenibles y saludables.

\section{Desafíos y Aprendizajes}
El desarrollo de una aplicación sostenible presenta varios desafíos:

\subsection{Optimización de Recursos:}
La optimización constante del uso de recursos puede ser compleja y requiere un análisis continuo y ajustes en el código y la infraestructura.

\subsection{Garantizar la Escalabilidad:}
Asegurar que la aplicación pueda escalar de manera eficiente sin comprometer la sostenibilidad fue un desafío significativo. Docker facilitó esta tarea, pero requirió un aprendizaje y adaptación constante.

\subsection{Mantener la Accesibilidad:}
Diseñar una aplicación accesible para todos los usuarios requiere un esfuerzo adicional en términos de diseño y pruebas, asegurando que la usabilidad no se vea comprometida.

\section{Conclusión}
En resumen, el proceso de desarrollar mi TFG sobre una aplicación web con Spring Boot y Docker para la gestión de cuestionarios sobre "soft skills" me ha proporcionado una comprensión profunda y multifacética de la sostenibilidad. He adquirido competencias clave en el desarrollo de software eficiente, la evaluación del impacto ambiental, la responsabilidad social y el uso de prácticas de código abierto. Estos aprendizajes son valiosos no solo para mi carrera profesional, sino también para mi desarrollo personal y mi capacidad de contribuir a un futuro más sostenible. La sostenibilidad en la tecnología es un reto continuo, y estoy comprometido a aplicar estos conocimientos y habilidades en todas las facetas de mi vida profesional y personal.
