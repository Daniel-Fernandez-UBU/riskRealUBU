\capitulo{4}{Técnicas y herramientas}


En esta parte de la memoria se van a presentar las distintas herramientas y técnicas utilizadas en el desarollo del proyecto.
Se incluirán todas las que se hayan valorado o utilizado, incluso aquellas que tras ciertas pruebas o pasos, se hayan descartado por haber encontrado una alternativa más funcional.

\section{Eclipse}

Herramienta de desarrollo (IDE) que se utilizará para llevar a cambo el proyecto.
A pesar de haber otras similares, como InteliJ, he preferido utilizar Eclipse al estar más fmailiarizado con ella por ser la utilizada en otras asignaturas de la carrera para la programación en Java.

\section{\LaTeX }
El editor de texto para generar toda la documentación relacionada con el trabajo de fin de grado será \LaTeX.
Aún estándo acostumbrado a utilizar Microsoft Word para este tipo de tareas, veo en utilizar \LaTeX  una oportunidad de aprender un nuevo generador de documentos.
La curva de aprendizaje es más grande que con Word (en mi caso), pero el resultado merece la pena porque permite aplicar formatos de forma sencilla, un aspecto en el que Word para documentos muy grandes, está en clara desventaja.

\section{Mik\TeX{}}
Para trabajar con plantillas de \LaTeX  en entornos windows existen varias alternativas, en mi caso he preferido utilizar el programa Mik\TeX{}.
Es un programa sencillo y tiene por separado tanto la parte de edición como la de presentación, para ir viendo los cambios una vez se compila la plantilla completa.

\section{Spring Boot}

Para la aplicación web en java se utilizará el Framework de Spring.
Es mi primera vez con un framework en java, y para aprender conceptos de forma más rápida y con ejemplos he tomado como referencia un curso en Udemy \cite{udemy:eliseo} para dar un acelerón en el aprendizaje.
Gracias a las dependencias predefinidas, facilita el desarrollo de la aplicación de forma considerable.

\subsection{Spring Web}

Es la dependencia principal de Spring Boot para la creación de aplicaciones web. 
Simplifica la creación de controladores, el manejo de las solicitude y se integra de forma sencilla con motores de plantillas para la creación de HTML dinámico.

\subsection{Spring JPA}

Es la dependencia de Spring para el manejo de entidades persistentes de una base de datos. 

Con los respositorios CRUD o JPA, y la inclusió por defecto de Hibernate, simplifica hasta el extremo la gestión de nuestros "objetos" en la base de datos.

\subsection{Spring Security}

Es la dependencia de Spring para dotar de seguridad nuestra aplicación de forma automática.

Esta seguridad se basa en autenticación y autorizació principalmente, para poder permitir ciertas partes de nuestra web a usuarios "invitados" o limitar el acceso a otras en función del rol del usuario que haya iniciado sesión.

Integra de forma predeterminada protección contra amenazas comunes, como CSRF (Cross-Site Request Forgery).

\section{Thymeleaf}

Es un motor de plantillas para html, para permitir la creación de html de forma dinámica.

Es un motor sencillo de usar, con expresiones básicas, que nos permite utilizar los objetos que enviamos para generar más o menos código, mostrar u ocultar opciones, \dots

\section{HTML}

Lenguaje utilizado en la creación de las páginas web, en particular HTML5, que tiene algunas funcionalidades básicas ya implementadas, como por ejemplo el uso de "required" en campos de formulario para que sea obligatorio rellenar los campos si necesidad de utilizar javascript para el control.

\section{MySQL}

Se utiliza MySQL como base de datos para almacenar los usuarios y roles.

Se ha decidido usar una base de datos porque su integración con Spring Security es muy sencilla y funcional. 

A través de los repositorios CRUD que nos proporciona Spring JPA podemos gestionar las tablas y registros de forma ágil.

\section{Docker}

\textbf{Docker} consiste en tener microservicios corriendo en una máquina, si necesidad de tener una gran infraestructura montada.

He utilizado docker tanto para el servidor de MySQL como para la propia aplicación Web, de esta forma con 2 simples docker, un docker-compose (fichero de configuración de dockers que permite descargarse, crear, parar o lanzar dockers) podemos tener toda la aplicación completa corriendo en cualquier entorno, ya que docker funciona en los principales sistemas operativos del mercado: Linux, Windows y macOS.

\section{GitHub}

Se va a utilizar GitHub para dar visibilidad al trabajo diario y constante en el proyecto, ya que permite realizar aportaciones incrementales de código, documentación, etc del TFG completo.

\subsection{Repository}

Se ha creado un repositorio~\cite{github:repo} donde se subirá todo el TFG al completo, tanto la documentación de la memoria, como la aplicación de java.

\subsection{Proyect}

Se ha creado un proyecto~\cite{github:proyect} relacionado, que sigue como la base la metodología Kanban, para crear nuevas tareas, relacionarlas con el repositorio y utilizar pizarras y paneles para ver el progreso, en lo que se está trabajando y lo ya completado.

\textbf{Se prescinde de esta herramienta al sustituirse por Zube.io}

\section{JSON}

Se van a utilizar ficheros con la estructura de un json para almacenar los cuestionarios completos, a los que se accederá desde la aplicación web de Java.

\section{Zube.io}

Se va a utilizar la versión gratuita de Zube.io, que permite generar Sprints, y gráficos de seguimiento, como Burndowns o Burnups.

Gracias a la sencilla y completa integración con Github ofrece una versión más completa de la parte de Proyect de Github, por lo que he decidido utilizar esta plataforma para el seguimiento de las tareas en vez de la que ofrecía Github.

La versión gratuita permite integrar lo necesario del repositorio y del proyecto, por lo que no es necesario ampliar a versiones más completas.







