\apendice{Plan de Proyecto Software}

\section{Introducción}

La fase de planificación es parte fundamental y necesaria en todos los proyectos.

Esta fase sive para ir evaluando la viabilidad del proyecto a lo largo de su vida, pudiendo prevenir o adelantarse a posibles inconvenientes que surgan durante la fase de desarrollo.
Tener medidas de forma correcta las tareas que se van realizando y su implicación en el total del proyecto es fundamental para un desarrollo sostenible, tanto económica como temporalmente.

Se ha decidido dividir esta fase en dos etapas:

\begin{itemize}
	\item
	Planificación temporal.
	\item
	Estudio de viabilidad.
\end{itemize}

La planificación temporal tratará sobre la gestión y aprovechamiento del tiempo, así como la división del proyecto en pequeños hitos, siguiendo una metodología ágil.

El estudio de viabilidad se descompone a su vez en:

\begin{itemize}
	\item
	Viabilidad económica: Inversión necesaria para acometer el proyecto y posible obtención de beneficios.
	\item
	Viabilidad legal: Todos los programas software conyevan unas implicaciones legales.
\end{itemize}

\section{Planificación temporal}

Para llevar a cabo una correcta planificación temporal se gestionado el proyecto siguiendo la metodología ágil que plantea Scrum, (poner enlace y bibliografía a Scrum) de forma "reducida" ya que el equipo de desarrolladores solo ha tenido un integrante, y las labores de "Scrum Master" y "Product Owner" han recaído en el tutor del TFG, el Sr. Raúl Marticorena.

Para el seguimiento del desarrollo incremental del proyecto software se ha divido la planificación en una serie de Sprints (enlace y referencia).

Para la visualización de los sprint se tenía intención de incluir los gráficos de "Burndown" pero debido a que han sido unos sprint vivos, donde se han inco incluyendo o desglosando tareas más genéricas en partes más específicas, se ha decidido utilizar los gráficos "Burnup" que representan de forma más real el número total de tareas completadas en cada sprint.

\subsection{Sprint 1 - Base del proyecto}
\textit{Del 27/02/2024 al 11/03/2024 - 20 horas}

Se estima una dedicación total de 20 horas de trabajo.
Se cumple con la estimación inicial de horas para completar todas las tareas.

Temas tratados durante el Sprint:

\begin{itemize}
	\item
	Uso e integración de zube.io con GitHub.
	\item
	Registro en https://app.riskreal.eu/ para analizar la parte "privada" de la aplicación.
	\item
	Schema json:
	\begin{itemize}
		\item
		Posibilidad de imagen en cada respuesta
		\item
		Atributo idioma en el cuestionario
	\end{itemize}
	Empezar a valorar y tener en cuenta la posibilidad de cuestionarios "Multi-idioma".
	\item
	Investigar sobre Thumeleaf y Spring Tools.
	\item
	Cargar un json de prueba en la app.
	\item
	Representación en una web del json.
	\item
	Posible estructura de directorios de la aplicación.
	\item
	Abordar la generación dinámica del contenido respecto al json.
\end{itemize}

Tareas tratadas cerradas a lo largo del sprint:

\imagen{Sprint1_issues}{Sprint 1 - Base del proyecto - Tareas}

Gráfico de Burnup del sprint:

\imagen{Sprint1_burnup}{Sprint 1 - Base del proyecto - Gráfico}



\subsection{Sprint 2}
\textit{Del 11/03/2024 al 25/03/2024 - 20 horas}

Se estima una dedicación total de 20 horas de trabajo.
Se cumple con la estimación inicial de horas para completar todas las tareas.

Temas tratados durante el Sprint:

\begin{itemize}
	\item
	Inclusión de más preguntas y respuestas en el cuestionario.
	\item
	Tratar las preguntas como independientes en la web (permitir transición entre preguntas).
	\item
	Recoger y mostrar valoraciones al terminar el cuestionario.
	\item
	Concepto de sesión web, para que no se mezclen datos de 2 usuarios.
	\item
	Primer acercamiento a un framework de estilo, bootstrap o similar.
\end{itemize}

Tareas tratadas cerradas a lo largo del sprint:

\imagen{Sprint2_issues}{Sprint 2 - Base del proyecto - Tareas}

Gráfico de Burnup del sprint:

\imagen{Sprint2_burnup}{Sprint 2 - Base del proyecto - Gráfico}


\subsection{Sprint 3}
\textit{Del 11/03/2024 al 25/03/2024 - 20 horas}

Se estima una dedicación total de 20 horas de trabajo.
Se cumple con la estimación inicial de horas para completar todas las tareas.

Temas tratados durante el Sprint:
\begin{itemize}
	\item
	Independizar las sesiones de los usuarios.
	\item
	Integración de la seguridad en la aplicación.
	\item
	Integrar la funcionalidad de envío de correo electrónico desde la web.
	\item
	Enviar la información del cuestionario con el método POST para que no vayan los datos en claro.
	\item
	Internacionalización de los distintos menús de la app web.
	\item
	Integración de framework CSS para utilizar un diseño "responsive" en la web.
	\item
	Recoger datos de puntuación de las preguntas.
	\item
	Mostrar un score tras finalizar el cuestionario.
\end{itemize}


Tareas tratadas cerradas a lo largo del sprint:

\imagen{Sprint3_issues}{Sprint 3 - Base del proyecto - Tareas}
\textit{Del 11/03/2024 al 25/03/2024 - 20 horas}

Se estima una dedicación total de 20 horas de trabajo.
Se cumple con la estimación inicial de horas para completar todas las tareas.

Gráfico de Burnup del sprint:

\imagen{Sprint3_burnup}{Sprint 3 - Base del proyecto - Gráfico}

\subsection{Sprint 4 - Evolución y finalización de la parte del cuestionario}


Tareas tratadas cerradas a lo largo del sprint:

\imagen{Sprint4_issues}{Sprint 4 - Base del proyecto - Tareas}

Gráfico de Burnup del sprint:

\imagen{Sprint4_burnup}{Sprint 4 - Evolución y finalización de la parte del cuestionario - Gráfico}

\subsection{Sprint 5 - Evolución y finalización de la parte del cuestionario 2}
\textit{Del 06/05/2024 al 22/05/2024 - 40 horas}

Se estima una dedicación total de 40 horas de trabajo.
Se cumple con la estimación inicial de horas para completar todas las tareas.
En este sprint se han completado las tareas que quedaron pendientes del sprint 3 y 4, por eso la carga de trabajo y de horas ha sido mayor que en los anteriores.

Temas tratados durante el Sprint:

\begin{itemize}
	\item
	Investigar sitio para alojar un test de la aplicación --> https://www.heroku.com/.
	Guardar el score por pregunta y el total.
	\item
	Descarga de resultados en CSV.
	\item
	Array de imágenes en las preguntas.
	\item
	Diseño de datos bbdd y json. Código de la aplicación.
	\item
	Revisar información para incluir en el arquitectónico.
	\item
	Revisar diagramas para incluir en el diseño procedimental.
\end{itemize}

Tareas tratadas cerradas a lo largo del sprint:

\imagen{Sprint5_issues}{Sprint 5 - Evolución y finalización de la parte del cuestionario 2 - Tareas}

Gráfico de Burnup del sprint:

\imagen{Sprint5_burnup}{Sprint 5 - Base del proyecto - Gráfico}

\subsection{Sprint 6}


\section{Estudio de viabilidad}

\subsection{Viabilidad económica}

\subsection{Viabilidad legal}


