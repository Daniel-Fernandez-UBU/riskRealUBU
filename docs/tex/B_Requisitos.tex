\apendice{Especificación de Requisitos}

\section{Introducción}

En este anexo se recoge la especificación de requisitos de la aplicación web que se ha desarrollado.

Este documento tiene el objetivo doble de servir de \textit{contrato} con el cliente y como base de documentación para el análisis de la aplicación.

\section{Objetivos generales}

El proyecto tiene los siguientes objetivos generales:

\begin{itemize}
	\item Desarrollar una aplicación web que permita la realización de cuestionarios para evaluar \textit{Soft Skills}.
	\item Permitir la carga de cuestionarios de forma dinámica, tanto a través de una carpeta como desde la propia interfaz web.
	\item Almacenar los resultados de los cuestionarios de forma anónima con fines estadísticos para los creadores.
	\item La descarga completa de los resultados en un documento correctamente estructurado y que sea sencillo de modelar.
	\item La aplicación web tiene que tener un diseño \textit{responsive} que extienda su funcionalidad a diferentes dispositivos.
	\item La aplicación tiene que tener ser multi-idioma, para poder llegar a más usuarios.
\end{itemize}

\section{Catálogo de requisitos}

Se van a enumerar los requisitos separándolos en dos categorías: requisitos funcionales y no funcionales \cite{web:requisitos}.

\subsection{Requisitos funcionales}
\begin{itemize}
	\item \textbf{RF-1 Gestión de cuestionarios}: La aplicación web tiene que ser capaz de gestionar cuestionarios.
	\begin{itemize}
		\item \textbf{RF-1.1 Carga de cuestionario}: Se tienen que poder cargar nuevos cuestionarios en la aplicación web.
		\item \textbf{RF-1.2 Generación de cuestionario}: El administrador debe poder generar cuestionarios a partir de los archivos json cargados.
		\item \textbf{RF-1.3 Realización de cuestionario}: Los usuarios deben poder realizar los cuestionarios disponibles.
	\end{itemize}
	\item \textbf{RF-2 Gestión de usuarios}: La aplicación web tiene que ser capaz de gestionar usuarios.
		\begin{itemize}
		\item \textbf{RF-2.1 Registro de usuario}: Los usuarios deben poder registrarse en la aplicación.
		\item \textbf{RF-2.2 Modificación de datos}: Los usuarios deben poder editar sus datos.
	\end{itemize}
	\item \textbf{RF-3 Gestión de resultados}: La apicación web tiene que permitir la gestión de los resultados obtenidos.
		\begin{itemize}
		\item \textbf{RF-3.1 Almacenamiento de resultados}: Los resultados se tienen que almacenar de forma persistente en la aplicación.
		\item \textbf{RF-3.2 Descarga de resultados}: Los resultatos se tienen que poder descargar de la aplicación.
	\end{itemize}
	\item \textbf{RF-4 Diseño internacional}: La aplicación web tiene que ser capaz de adaptarse a varios idiomas.
		\begin{itemize}
		\item \textbf{RF-4.1 Selección de idioma}: La aplicación debe permitir la selección del idioma.
		\item \textbf{RF-4.2 Idioma de los cuestionarios}: La aplicación debe permitir la carga de cuestionarios en varios idiomas.
	\end{itemize}
\end{itemize}

\subsection{Requisitos no funcionales}
\begin{itemize}
	\item \textbf{RNF-1 Usabilidad}: La aplicación debe ser intuitiva, con una curva de aprendizaje baja.
	\item \textbf{RNF-2 Rendimiento}: La aplicación web tiene que tener unos tiempo de carga y respuesta bajos.
	\item \textbf{RNF-3 Escalabilidad}: La aplicación web tiene que poder funcionar en cualquier navegador, y permitir añadir funcionalidades de forma sencilla.
	\item \textbf{RNF-4 Portabilidad}: La aplicación web tiene que adaptarse a diferentes dispositivos.
	\item \textbf{RNF-5 Internacionalizaicón}: La aplicación web tiene que estar preparada para soportar varios idiomas.
	\item \textbf{RNF-6 Seguridad}: La aplicación web debe gestionar los accesos y datos sensibles de forma adecuada.
\end{itemize}

\section{Especificación de requisitos}

Esta sección va a incluir el diagrama de casos de uso y la explicación de cada uno.

\begin{landscape}
\subsection{Diagrama de casos de uso}
\imagenAncho{diagramaCasosUso}{Diagrama de casos de uso.}{1.35}
\end{landscape}

\subsection{Actores}

Pueden interactuar con el sistema 3 actores diferentes:

\begin{itemize}
	\item Usuario invitado: Usuario que accede a la página de forma anómina, y puede probar los cuestionarios de ejemplo o registrarse.  
	\item Usuario registrado: Usuario que accede a la página e inicia sesión. Puede ver su perfil y modificarlo o realizar los cuestionarios disponibles.
	\item Administrador: Usuario administrador que puede gestionar roles de los usuarios registrados, descargarse los resultados completos de todos los cuestionarios realizados, cargar y generar más cuestionarios, o realizar alguno de los cuestionarios disponibles.
\end{itemize}

\begin{table}[p]
	\centering
	\begin{tabularx}{\linewidth}{ p{0.21\columnwidth} p{0.71\columnwidth} }
		\toprule
		\textbf{CU-01}    & \textbf{Cargar Cuestionarios}\\
		\toprule
		\textbf{Versión}              & 1.0    \\
		\textbf{Autor}                & Daniel Fernández Barrientos \\
		\textbf{Requisitos asociados} & RF-1, RF-1.1 \\
		\textbf{Descripción}          & Permite al administrador cargar nuevos cuestionarios. \\
		\textbf{Precondición}         & La aplicacion se encuentra levantada, y ha iniciado sesión un usuario administrador. \\
		\textbf{Acciones}             &
		\begin{enumerate}
			\def\labelenumi{\arabic{enumi}.}
			\tightlist
			\item El usuario administrador inicia sesión.
			\item Accede al apartador de carga de cuestionarios.
			\item Selecciona un archivo json con una estructura válida y lo carga en la aplicación.
		\end{enumerate}\\
		\textbf{Postcondición}        & El nuevo archivo json se encuentra disponible para generar un nuevo cuestionario. \\
		\textbf{Excepciones}          & Error si el archivo seleccionado no es un json o si su estructura no es la esperada. \\
		\textbf{Importancia}          & Alta. \\
		\bottomrule
	\end{tabularx}
	\caption{CU-01 Cargar Cuestionarios.}
\end{table}

\begin{table}[p]
	\centering
	\begin{tabularx}{\linewidth}{ p{0.21\columnwidth} p{0.71\columnwidth} }
		\toprule
		\textbf{CU-02}    & \textbf{Generar Cuestionarios}\\
		\toprule
		\textbf{Versión}              & 1.0    \\
		\textbf{Autor}                & Daniel Fernández Barrientos \\
		\textbf{Requisitos asociados} & RF-1, RF-1.2 \\
		\textbf{Descripción}          & Permite al administrador generar nuevos cuestionarios. \\
		\textbf{Precondición}         & La aplicacion se encuentra levantada, ha iniciado sesión un usuario administrador y hay cuestionarios pendientes de generar. \\
		\textbf{Acciones}             &
		\begin{enumerate}
			\def\labelenumi{\arabic{enumi}.}
			\tightlist
			\item El usuario administrador inicia sesión.
			\item Accede al apartador de carga de cuestionarios.
			\item Selecciona uno de los archivos json cargados y genera un nuevo cuestionario.
		\end{enumerate}\\
		\textbf{Postcondición}        & Hay un nuevo cuestionario disponible en la aplicación.. \\
		\textbf{Excepciones}          & Error si el archivo json utilizado no tiene una estructura correcta. \\
		\textbf{Importancia}          & Alta. \\
		\bottomrule
	\end{tabularx}
	\caption{CU-02 Generar Cuestionarios.}
\end{table}

\begin{table}[p]
	\centering
	\begin{tabularx}{\linewidth}{ p{0.21\columnwidth} p{0.71\columnwidth} }
		\toprule
		\textbf{CU-03}    & \textbf{Realizar Cuestionario}\\
		\toprule
		\textbf{Versión}              & 1.0    \\
		\textbf{Autor}                & Daniel Fernández Barrientos \\
		\textbf{Requisitos asociados} & RF-1.3, RF-3, RF-3.1 \\
		\textbf{Descripción}          & Permite a cualquier usuario realizar cuestionarios. \\
		\textbf{Precondición}         & La aplicacion se encuentra levantada y hay cuestionarios disponibles en el idioma seleccionado. \\
		\textbf{Acciones}             &
		\begin{enumerate}
			\def\labelenumi{\arabic{enumi}.}
			\tightlist
			\item Un usuario accede a la página web.
			\item Selecciona su idioma o mantiene el por defecto.
			\item Accede a uno de los cuestionarios que se le presentan.
			\item Contesta a las preguntas y llega hasta el final, donde termina el cuestionario.
			\item Ve una página con el resultado obtenido.
		\end{enumerate}\\
		\textbf{Postcondición}        & Se ha incluido una línea nueva en el csv de resultados. \\
		\textbf{Excepciones}          & Alguna de las imágenes del cuestionario no está presente. \\
		\textbf{Importancia}          & Alta. \\
		\bottomrule
	\end{tabularx}
	\caption{CU-03 Realizar Cuestionario.}
\end{table}

\begin{table}[p]
	\centering
	\begin{tabularx}{\linewidth}{ p{0.21\columnwidth} p{0.71\columnwidth} }
		\toprule
		\textbf{CU-04}    & \textbf{Registro de Usuarios}\\
		\toprule
		\textbf{Versión}              & 1.0    \\
		\textbf{Autor}                & Daniel Fernández Barrientos \\
		\textbf{Requisitos asociados} & RF-2, RF-2.1 \\
		\textbf{Descripción}          & Permite a cualquier usuario registrarse en la aplicación web. \\
		\textbf{Precondición}         & La aplicacion se encuentra levantada junto con la base de datos. \\
		\textbf{Acciones}             &
		\begin{enumerate}
			\def\labelenumi{\arabic{enumi}.}
			\tightlist
			\item Un usuario accede a la página web.
			\item Selecciona el registro.
			\item Completa el formulario y se registra.
		\end{enumerate}\\
		\textbf{Postcondición}        & Se ha añadido un nuevo usuario a la aplicación. \\
		\textbf{Importancia}          & Alta. \\
		\bottomrule
	\end{tabularx}
	\caption{CU-04 Registro de Usuarios.}
\end{table}

\begin{table}[p]
	\centering
	\begin{tabularx}{\linewidth}{ p{0.21\columnwidth} p{0.71\columnwidth} }
		\toprule
		\textbf{CU-05}    & \textbf{Iniciar Sesión en la web}\\
		\toprule
		\textbf{Versión}              & 1.0    \\
		\textbf{Autor}                & Daniel Fernández Barrientos \\
		\textbf{Requisitos asociados} & RF-2 \\
		\textbf{Descripción}          & Los usuarios registrados pueden iniciar sesión. \\
		\textbf{Precondición}         & La aplicacion se encuentra levantada y el usuario está registrado. \\
		\textbf{Acciones}             &
		\begin{enumerate}
			\def\labelenumi{\arabic{enumi}.}
			\tightlist
			\item Un usuario accede a la página web.
			\item Selecciona la opción de iniciar sesión.
			\item Introduce sus datos de usuario.
			\item Si son correctos, accede a la parte privada de la aplicación.
			\item Si no lo son, se permiten más intentos o la posibilidad de recibir un correo con una contraseña nueva de un solo uso.
			\item Se obliga al cambio tras introducir la contraseña recibida por correo.
		\end{enumerate}\\
		\textbf{Postcondición}        & Puede acceder a todos los cuestionarios disponibles y ver o actualizar su perfil. \\
		\textbf{Excepciones}          & Usuario o contraseña incorrectos. \\
		\textbf{Importancia}          & Alta. \\
		\bottomrule
	\end{tabularx}
	\caption{CU-05 Iniciar Sesión en la web.}
\end{table}

\begin{table}[p]
	\centering
	\begin{tabularx}{\linewidth}{ p{0.21\columnwidth} p{0.71\columnwidth} }
		\toprule
		\textbf{CU-6}    & \textbf{Modificar datos del perfil}\\
		\toprule
		\textbf{Versión}              & 1.0    \\
		\textbf{Autor}                & Daniel Fernández Barrientos \\
		\textbf{Requisitos asociados} & RF-2, RF-2.2 \\
		\textbf{Descripción}          & Permite a cualquier usuario identificado modificar su perfil. \\
		\textbf{Precondición}         & El usuario ha iniciado sesión. \\
		\textbf{Acciones}             &
		\begin{enumerate}
			\def\labelenumi{\arabic{enumi}.}
			\tightlist
			\item Un usuario inicia sesión en la página web.
			\item Accede a su perfil.
			\item Modifica los datos que sean erróneos (el nombre de usuario no permite edición).
			\item Pulsa en actualizar y se muestra el perfil actualizado.
		\end{enumerate}\\
		\textbf{Postcondición}        & Se han actualizado sus datos de perfil. \\
		\textbf{Importancia}          & Alta. \\
		\bottomrule
	\end{tabularx}
	\caption{CU-06 Modificar datos del perfil.}
\end{table}

\begin{table}[p]
	\centering
	\begin{tabularx}{\linewidth}{ p{0.21\columnwidth} p{0.71\columnwidth} }
		\toprule
		\textbf{CU-07}    & \textbf{Gestión de Roles}\\
		\toprule
		\textbf{Versión}              & 1.0    \\
		\textbf{Autor}                & Daniel Fernández Barrientos \\
		\textbf{Requisitos asociados} & RF-2, RF-2.2 \\
		\textbf{Descripción}          & Permite a un administrador actualizar el rol en la aplicación de otros usuarios. \\
		\textbf{Precondición}         & La aplicación está levantada y hay usuarios registrados en la base de datos. \\
		\textbf{Acciones}             &
		\begin{enumerate}
			\def\labelenumi{\arabic{enumi}.}
			\tightlist
			\item Un administrador inicia sesión en la web.
			\item Selecciona la opción de gestión de roles.
			\item Modifica el perfil de seguridad de los usuarios que considere.
			\item Pulsa en actualizar y se muestran de nuevos los perfiles actualizados.
		\end{enumerate}\\
		\textbf{Postcondición}        & Se ha actualizado el perfil de seguridad de los usuarios correspondientes. \\
		\textbf{Importancia}          & Media. \\
		\bottomrule
	\end{tabularx}
	\caption{CU-07. Gestión de Roles.}
\end{table}

\begin{table}[p]
	\centering
	\begin{tabularx}{\linewidth}{ p{0.21\columnwidth} p{0.71\columnwidth} }
		\toprule
		\textbf{CU-08}    & \textbf{Descarga de Resultados}\\
		\toprule
		\textbf{Versión}              & 1.0    \\
		\textbf{Autor}                & Daniel Fernández Barrientos \\
		\textbf{Requisitos asociados} & RF-3, RF-3.2 \\
		\textbf{Descripción}          & Permite a un administrador descargarse todos los resultados de los cuestionarios que se han realizado. \\
		\textbf{Precondición}         & La aplicación se encuentra levantada y ha iniciado sesión un administrador. \\
		\textbf{Acciones}             &
		\begin{enumerate}
			\def\labelenumi{\arabic{enumi}.}
			\tightlist
			\item Un administrador accede a la web.
			\item Selecciona la opción de descarga de resultados.
			\item Se descarga un csv con todos los resultados disponibles.
		\end{enumerate}\\
		\textbf{Postcondición}        & Se ha descargado un csv en el equipo del usuario administrador. \\
		\textbf{Excepciones}          & Si no se ha realizado ningún cuestionario, el csv estará vacío. \\
		\textbf{Importancia}          & Alta. \\
		\bottomrule
	\end{tabularx}
	\caption{CU-08 Descarga de Resultados.}
\end{table}

\begin{table}[p]
	\centering
	\begin{tabularx}{\linewidth}{ p{0.21\columnwidth} p{0.71\columnwidth} }
		\toprule
		\textbf{CU-09}    & \textbf{Selección de Idioma}\\
		\toprule
		\textbf{Versión}              & 1.0    \\
		\textbf{Autor}                & Daniel Fernández Barrientos \\
		\textbf{Requisitos asociados} & RF-4, RF-4.1. RF-4.2 \\
		\textbf{Descripción}          & Permite realizar cuestionarios o ver la aplicación en diferentes idiomas. \\
		\textbf{Precondición}         & La aplicación se encuentra levantada y hay cuestionarios generados en varios idiomas. \\
		\textbf{Acciones}             &
		\begin{enumerate}
			\def\labelenumi{\arabic{enumi}.}
			\tightlist
			\item Un usuario accede a la página web.
			\item Selecciona su idioma o mantiene el por defecto.
			\item Los menús de la web aparecen en el idioma seleccionado.
			\item Los cuestionarios aparecen en el idioma seleccionado.
		\end{enumerate}\\
		\textbf{Postcondición}        & La aplicación se muestra en el idioma seleccionado y le actualiza la lista de cuestionarios disponibles. \\
		\textbf{Excepciones}          & No hay cuestionarios diponibles en algún idioma. \\
		\textbf{Importancia}          & Alta. \\
		\bottomrule
	\end{tabularx}
	\caption{CU-09 Selección de Idioma.}
\end{table}


