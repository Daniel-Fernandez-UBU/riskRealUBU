\capitulo{3}{Conceptos teóricos}

\section{Introducción}

Las habilidades blandas son esenciales en todos los ámbitos de la vida.

\begin{itemize}
	\item Personal: Saber comunicarte y relacionarte de forma efectiva con tus familiares o amigos es la base de las relaciones humanas.
	\item Laboral: Tener la capacidad de liderar o de expresar problemas, saber cómo tratar a las personas que tienes a tu cargo, es un pilar fundamental de un buen ambiente de trabajo.
\end{itemize}

Norlmamente solo somos capaces de enumerar habilidades blandas relacionadas con el trabajo:

\begin{enumerate}
	\item Liderazgo.
	\item Trabajo en equipo.
	\item	Toma de decisiones.
	\item Creatividad.
	\item Adaptabilidad.
\end{enumerate}

Pero todas las habilidades son aplicables a cualquier situación, no porque laboralmente seamos creativos, significa que en nuestro entorno personal no tenemos que serlo.

Todos poseemos ese tipo de habilidades, en mayor o menor grado, pero hasta que no se le da un valor \textit{numérico} a una cualidad, no la tomamos en consideración.	

\section{Importancia de las habilidades blandas a lo largo de la historia}

Aunque pensemos que las habilidades blandas es \textit{algo moderno que se han inventado los psicólogos de ahora para tenernos todo el día haciendo cuestionarios}, son esas habilidades las que han ido marcando los hitos de la historia.

\begin{enumerate}
	\item Antigua Roma: La persuasión, el liderazgo o la toma de decisiones, eran fundamentales en la base de la civilización romana. La extensión del Imperio Romano es un ejemplo de un gran liderazgo y del saber tomar las decisiones correctas en cada batalla o conflicto.
	\item Revolución industrial: Tras la industrialización, habilidades como la negociación, el trabajo en equipo o la adaptabilidad fueron determinantes.
	\item Época actual: Con la constante evolución de las tecnologías y las formas de comunicarse, tras la aparición de Internet, el \textit{pozo de sabiduría del que todo el mundo puede beber}, disponer de ciertas habilidades sociales más allá de un cierto nivel de habilidades técnicas, es un factor diferenciador.
\end{enumerate}

\section{Las habilidades blandas en la actualidad}

No hace tanto tiempo, no se escuchaba nada sobre este tipo de habilidades, lo único que importaba era tener un título y cierto número de idiomas y con eso el trabajo y un puesto de responsabilidad estaban asegurados.

La mayoría de las empresas ~\cite{doc:skills} han ido evolucionando hasta sistemas de contratación donde se comprueban las habilidades de los candidatos a seleccionar.

Pero para poder conocer qué candidato es el realmente idóneo para un puesto, es necesario conocer el entorno de ese puesto de trabajo.

Las evaluaciones 360º ~\cite{doc:eva360} no solo miden el desempeño en el trabajo, sirven para medir las habilidades que tiene cada trabajador, y lo que ven el resto de compañeros o responsables a lo largo del año.

Las habilidades blandas van a marcar la ruta que se deberá seguir para crecer en el mundo digital que se está creando a nuestro alrededor, conocer las debilidades y potenciarlas, y las fortalezas y mantenerlas, supondrá un antes y un después en las relaciones interpersonales.



