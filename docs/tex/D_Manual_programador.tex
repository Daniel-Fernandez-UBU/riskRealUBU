\apendice{Documentación técnica de programación}

\section{Introducción}

En el anexo actual se va a describir toda la documentación técnica, así como la estructura de directorios que se ha seguido para almacenar el proyecto completo.
Una de las partes más importantes es la sección de \textit{Manual del Programador} ya que en ella se describen los pasos necesarios para instalar el entorno de desarrollo. 

\section{Estructura de directorios}

El repositorio del proyecto está formado por los siguientes directorios:

\section{Manual del programador}

El manual del programador es la guía \textit{paso a paso} de las diferentes herramientas, configuraciones y peculiaridades que se deben seguir para poder montar el entorno de desarrollo, trabajar con el código fuente o recompilar una nueva versión de la aplicación.

Las imágenes que se pueden ver en los siguientes pasos han sido tomadas desde un portátil con el sistema operativo Windows 11 y con las versiones aquí indicadas de las diferentes aplicaciones, por lo tanto, en distintas versiones el contenido o la ubicación de algunas opciones pueden estar en diferentes apartados.

\subsection{Herramientas necesarias}

Para poder desplegar el entorno de desarrollo se necesita instalar las siguientes aplicaciones:

\begin{itemize}
	\item Java SE JDK 17.
	\item Docker.
	\item Git.
	\item Spring Tools Suite 4.
\end{itemize}

A contunación, se explica cómo instalar y configurar cada uno de ellos de forma correcta.

\subsection{Java SE JDK 17}

Es el lenguaje de programación utilizado para la realización de la aplicación web.
Se debe acceder al siguiente enlace \cite{web:JavaJDK}, seleccionar el sistema operativo y arquitectura correspondiente y seguir el asistente de instalación.

\subsection{Docker}

En el momento actual en el que se encuentra la informática y el interés por la optimización de recursos, Docker se ha posicionado como la plataforma principal de \textit{microservicios}.

Docker Desktop se puede descargar desde el enlace \cite{web:dockerDesktop}. Tiene versión para windows, linux y mac, por lo que hay que elegir el sistema operativo correcto y seguir los pasos del asistente de instalación.

Una vez instalado, la aplicación sería similar a la siguiente:
\imagen{dockerDesktopMain}{Pantalla principal de Docker Desktop.}

Podemos cerrar la aplicación sin problema porque se queda en ejecución en segundo plano, pero para confirmarlo, podemos abrir \textit{el símbolo del sistema - cmd} o una consola de powershell y ejecutar el comando \textbf{docker version}, donde deberíamos obtener una imagen similar a la siguiente:
\imagen{dockerVersion}{Versión de docker en ejecución desde CMD de Windows.}

Lo importante de la imagen anterior es que aparezca la información de \textbf{Server: Docker Desktop}, si no apareciese, abrir la aplicación de Docker Desktop y comprobar que ahora si tenemos la información del servidor.

\subsection{Git}

Git es la herramienta que nos permite recuperar todo el proyecto desde el repositorio web \cite{github:proyect}. 
Nos descargamos Git desde el siguiente enlace \cite{web:git}.
Seleccionamos la versión que se corresponda con nuestro sistema operativo y seguimos los pasos del asistente de instalación.

Una vez instalado, vamos a utilizar la herramienta \textbf{Git Bash}
\imagen{gitBash}{Aplicación Git Bash.}

Navegamos por los directorios hasta el directorio en el que queramos replicar el proyecto y a continuación, escribimos \textit{git clone https://github.com/Daniel-Fernandez-UBU/riskRealUBU}
\imagen{gitBash}{Comando Git Clone.}


\section{Compilación, instalación y ejecución del proyecto}

\section{Pruebas del sistema}
