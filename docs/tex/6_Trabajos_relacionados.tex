\capitulo{6}{Trabajos relacionados}

Sobre este proyecto en particular, solo hay un trabajo relacionado, la aplicación web App.RiskReal.eu, la cuál ha servido de base de requisitos para la realización del proyecto actual.

\section{Web RiskReal.eu}
Es la página web \cite{web:riskreal} que se ha utilizado como base para la creación del nuevo proyecto.
La aplicación se basa en la utilización de diferentes cuestionarios para evaluar ``soft skills'' de trabajadores.

\subsection{Descripción general}
Desde la página principal se ofrece una descripción básica de lo que se puede hacer.
Indica que con diversos escenarios de test, se pueden evaluar de forma eficiente las diferentes ``soft skills''.

Consta de dos páginas base, \textit{Inicio}, que se consideraría la parte \textit{abierta} de la web, y \textit{Cursos}, con acceso restringido solo para usuarios registrados.

\subsection{Parte abierta}
Permite realizar un cuestionario de ejemplo, para obtener una evaluación orientativa de en qué estado nos encontramos en cuanto a varias habilidades blandas; y un cuestionario con diferentes situaciones, donde el ``trabajador'' va contestando en función de cómo las afrontaría.

No permite retroceder en las preguntas, tan solo se puede avanzar.

El diseño de los tipos de cuestionario está diferenciado de inicio:

\begin{itemize}
	\item Cuestionario tipo test: Una pregunta con varias posibles respuestas, bien categóricas o binarias.
	\item Cuestionario tipo escenario: Una pregunta con varias imágenes a modo de``situaciones'' y varias respuestas, donde se tiene que elegir la situación que más se acerque a lo que cada uno haría.
\end{itemize}

El no poder mezclar todo tipo de preguntas es una limitación en cuanto a la variedad de cuestionarios a poder realizar.

Lo anterior se ha tenido en cuenta para que cada pregunta se gestione a nivel particular, no a nivel de cuestionario completo, pudiendo mezclar los diferentes tipos de preguntas en el mismo cuestionario.

\subsection{Parte privada}
Se realiza un intento de registro con los siguentes datos:
\imagen{dataRiskRealAppRegister}{Datos de registro de prueba.}{.5}
Tras ello, no se recibe ningún correo de confirmación ni se consigue acceder a la propia aplicación.
Debido al contratiempo anterior, no se puede analizar más en detalle como es la parte ``privada'' de la aplicación.
\clearpage
Tampoco se permite el registro en la zona de empresas, obteniendo el mensaje que se puede ver en la siguiente imagen:
\imagen{errorRiskRealAppRegister}{Error tras enviar el formulario de \textit{Ir a la zona para empresas}.}{.5}





