\documentclass[a4paper,12pt,twoside]{memoir}

% Castellano
\usepackage[spanish,es-tabla]{babel}
\selectlanguage{spanish}
\usepackage[utf8]{inputenc}
\usepackage[T1]{fontenc}
\usepackage{lmodern} % Scalable font
\usepackage{microtype}
\usepackage{placeins}

\RequirePackage{booktabs}
\RequirePackage[table]{xcolor}
\RequirePackage{xtab}
\RequirePackage{multirow}

% Links
\PassOptionsToPackage{hyphens}{url}\usepackage[colorlinks]{hyperref}
\hypersetup{
	allcolors = {red}
}

% Ecuaciones
\usepackage{amsmath}

% Rutas de fichero / paquete
\newcommand{\ruta}[1]{{\sffamily #1}}

% Párrafos
\nonzeroparskip

% Huérfanas y viudas
\widowpenalty100000
\clubpenalty100000

% Imágenes

% Comando para insertar una imagen en un lugar concreto.
% Los parámetros son:
% 1 --> Ruta absoluta/relativa de la figura
% 2 --> Texto a pie de figura
% 3 --> Tamaño en tanto por uno relativo al ancho de página
\usepackage{graphicx}
\newcommand{\imagen}[3]{
	\begin{figure}[!h]
		\centering
		\includegraphics[width=#3\textwidth]{#1}
		\caption{#2}\label{fig:#1}
	\end{figure}
	\FloatBarrier
}

% Comando para insertar una imagen sin posición.
% Los parámetros son:
% 1 --> Ruta absoluta/relativa de la figura
% 2 --> Texto a pie de figura
% 3 --> Tamaño en tanto por uno relativo al ancho de página
\newcommand{\imagenflotante}[3]{
	\begin{figure}
		\centering
		\includegraphics[width=#3\textwidth]{#1}
		\caption{#2}\label{fig:#1}
	\end{figure}
}

% El comando \figura nos permite insertar figuras comodamente, y utilizando
% siempre el mismo formato. Los parametros son:
% 1 --> Porcentaje del ancho de página que ocupará la figura (de 0 a 1)
% 2 --> Fichero de la imagen
% 3 --> Texto a pie de imagen
% 4 --> Etiqueta (label) para referencias
% 5 --> Opciones que queramos pasarle al \includegraphics
% 6 --> Opciones de posicionamiento a pasarle a \begin{figure}
\newcommand{\figuraConPosicion}[6]{%
  \setlength{\anchoFloat}{#1\textwidth}%
  \addtolength{\anchoFloat}{-4\fboxsep}%
  \setlength{\anchoFigura}{\anchoFloat}%
  \begin{figure}[#6]
    \begin{center}%
      \Ovalbox{%
        \begin{minipage}{\anchoFloat}%
          \begin{center}%
            \includegraphics[width=\anchoFigura,#5]{#2}%
            \caption{#3}%
            \label{#4}%
          \end{center}%
        \end{minipage}
      }%
    \end{center}%
  \end{figure}%
}

%
% Comando para incluir imágenes en formato apaisado (sin marco).
\newcommand{\figuraApaisadaSinMarco}[5]{%
  \begin{figure}%
    \begin{center}%
    \includegraphics[angle=90,height=#1\textheight,#5]{#2}%
    \caption{#3}%
    \label{#4}%
    \end{center}%
  \end{figure}%
}
% Para las tablas
\newcommand{\otoprule}{\midrule [\heavyrulewidth]}
%
% Nuevo comando para tablas pequeñas (menos de una página).
\newcommand{\tablaSmall}[5]{%
 \begin{table}
  \begin{center}
   \rowcolors {2}{gray!35}{}
   \begin{tabular}{#2}
    \toprule
    #4
    \otoprule
    #5
    \bottomrule
   \end{tabular}
   \caption{#1}
   \label{tabla:#3}
  \end{center}
 \end{table}
}

%
% Nuevo comando para tablas pequeñas (menos de una página).
\newcommand{\tablaSmallSinColores}[5]{%
 \begin{table}[H]
  \begin{center}
   \begin{tabular}{#2}
    \toprule
    #4
    \otoprule
    #5
    \bottomrule
   \end{tabular}
   \caption{#1}
   \label{tabla:#3}
  \end{center}
 \end{table}
}

\newcommand{\tablaApaisadaSmall}[5]{%
\begin{landscape}
  \begin{table}
   \begin{center}
    \rowcolors {2}{gray!35}{}
    \begin{tabular}{#2}
     \toprule
     #4
     \otoprule
     #5
     \bottomrule
    \end{tabular}
    \caption{#1}
    \label{tabla:#3}
   \end{center}
  \end{table}
\end{landscape}
}

%
% Nuevo comando para tablas grandes con cabecera y filas alternas coloreadas en gris.
\newcommand{\tabla}[6]{%
  \begin{center}
    \tablefirsthead{
      \toprule
      #5
      \otoprule
    }
    \tablehead{
      \multicolumn{#3}{l}{\small\sl continúa desde la página anterior}\\
      \toprule
      #5
      \otoprule
    }
    \tabletail{
      \hline
      \multicolumn{#3}{r}{\small\sl continúa en la página siguiente}\\
    }
    \tablelasttail{
      \hline
    }
    \bottomcaption{#1}
    \rowcolors {2}{gray!35}{}
    \begin{xtabular}{#2}
      #6
      \bottomrule
    \end{xtabular}
    \label{tabla:#4}
  \end{center}
}

%
% Nuevo comando para tablas grandes con cabecera.
\newcommand{\tablaSinColores}[6]{%
  \begin{center}
    \tablefirsthead{
      \toprule
      #5
      \otoprule
    }
    \tablehead{
      \multicolumn{#3}{l}{\small\sl continúa desde la página anterior}\\
      \toprule
      #5
      \otoprule
    }
    \tabletail{
      \hline
      \multicolumn{#3}{r}{\small\sl continúa en la página siguiente}\\
    }
    \tablelasttail{
      \hline
    }
    \bottomcaption{#1}
    \begin{xtabular}{#2}
      #6
      \bottomrule
    \end{xtabular}
    \label{tabla:#4}
  \end{center}
}

%
% Nuevo comando para tablas grandes sin cabecera.
\newcommand{\tablaSinCabecera}[5]{%
  \begin{center}
    \tablefirsthead{
      \toprule
    }
    \tablehead{
      \multicolumn{#3}{l}{\small\sl continúa desde la página anterior}\\
      \hline
    }
    \tabletail{
      \hline
      \multicolumn{#3}{r}{\small\sl continúa en la página siguiente}\\
    }
    \tablelasttail{
      \hline
    }
    \bottomcaption{#1}
  \begin{xtabular}{#2}
    #5
   \bottomrule
  \end{xtabular}
  \label{tabla:#4}
  \end{center}
}



\definecolor{cgoLight}{HTML}{EEEEEE}
\definecolor{cgoExtralight}{HTML}{FFFFFF}

%
% Nuevo comando para tablas grandes sin cabecera.
\newcommand{\tablaSinCabeceraConBandas}[5]{%
  \begin{center}
    \tablefirsthead{
      \toprule
    }
    \tablehead{
      \multicolumn{#3}{l}{\small\sl continúa desde la página anterior}\\
      \hline
    }
    \tabletail{
      \hline
      \multicolumn{#3}{r}{\small\sl continúa en la página siguiente}\\
    }
    \tablelasttail{
      \hline
    }
    \bottomcaption{#1}
    \rowcolors[]{1}{cgoExtralight}{cgoLight}

  \begin{xtabular}{#2}
    #5
   \bottomrule
  \end{xtabular}
  \label{tabla:#4}
  \end{center}
}



\graphicspath{ {./img/} }

% Capítulos
\chapterstyle{bianchi}
\newcommand{\capitulo}[2]{
	\setcounter{chapter}{#1}
	\setcounter{section}{0}
	\setcounter{figure}{0}
	\setcounter{table}{0}
	\chapter*{\thechapter.\enskip #2}
	\addcontentsline{toc}{chapter}{\thechapter.\enskip #2}
	\markboth{#2}{#2}
}

% Apéndices
\renewcommand{\appendixname}{Apéndice}
\renewcommand*\cftappendixname{\appendixname}

\newcommand{\apendice}[1]{
	%\renewcommand{\thechapter}{A}
	\chapter{#1}
}

\renewcommand*\cftappendixname{\appendixname\ }

% Formato de portada
\makeatletter
\usepackage{xcolor}
\newcommand{\tutor}[1]{\def\@tutor{#1}}
\newcommand{\course}[1]{\def\@course{#1}}
\definecolor{cpardoBox}{HTML}{E6E6FF}
\def\maketitle{
  \null
  \thispagestyle{empty}
  % Cabecera ----------------
\noindent\includegraphics[width=\textwidth]{cabecera}\vspace{1cm}%
  \vfill
  % Título proyecto y escudo informática ----------------
  \colorbox{cpardoBox}{%
    \begin{minipage}{.8\textwidth}
      \vspace{.5cm}\Large
      \begin{center}
      \textbf{TFG del Grado en Ingeniería Informática}\vspace{.6cm}\\
      \textbf{\LARGE\@title{}}
      \end{center}
      \vspace{.2cm}
    \end{minipage}

  }%
  \hfill\begin{minipage}{.20\textwidth}
    \includegraphics[width=\textwidth]{escudoInfor}
  \end{minipage}
  \vfill
  % Datos de alumno, curso y tutores ------------------
  \begin{center}%
  {%
    \noindent\LARGE
    Presentado por \@author{}\\ 
    en Universidad de Burgos --- \@date{}\\
    Tutor: \@tutor{}\\
  }%
  \end{center}%
  \null
  \cleardoublepage
  }
\makeatother

\newcommand{\nombre}{Daniel Fernández Barrientos} %%% cambio de comando
\newcommand{\dni}{51079809-Y} %%% cambio de comando
\newcommand{\tfgTitulo}{Generador de cuestionarios y escenarios de test sobre soft skills} %%% cambio de comando
\newcommand{\tfgTutor}{Raúl Marticorena Sanchez} %%% cambio de comando

% Datos de portada
\title{\tfgTitulo}
\author{\nombre}
\tutor{\tfgTutor}
\date{\today}

\begin{document}

\maketitle


\newpage\null\thispagestyle{empty}\newpage


%%%%%%%%%%%%%%%%%%%%%%%%%%%%%%%%%%%%%%%%%%%%%%%%%%%%%%%%%%%%%%%%%%%%%%%%%%%%%%%%%%%%%%%%
\thispagestyle{empty}


\noindent\includegraphics[width=\textwidth]{cabecera}\vspace{1cm}

\noindent D. \tfgTutor, profesor del departamento de nombre departamento, área de nombre área.

\noindent Expone:

\noindent Que el alumno D. \nombre, con DNI \dni, ha realizado el Trabajo final de Grado en Ingeniería Informática titulado \tfgTitulo de TFG. 

\noindent Y que dicho trabajo ha sido realizado por el alumno bajo la dirección del que suscribe, en virtud de lo cual se autoriza su presentación y defensa.

\begin{center} %\large
En Burgos, {\large \today}
\end{center}

\vfill\vfill\vfill

% Author and supervisor
\begin{minipage}{0.45\textwidth}
\begin{flushleft} %\large
Vº. Bº. del Tutor:\\[2cm]
D. nombre tutor
\end{flushleft}
\end{minipage}
\hfill
\begin{minipage}{0.45\textwidth}
\begin{flushleft} %\large
Vº. Bº. del co-tutor:\\[2cm]
D. nombre co-tutor
\end{flushleft}
\end{minipage}
\hfill

\vfill

% para casos con solo un tutor comentar lo anterior
% y descomentar lo siguiente
%Vº. Bº. del Tutor:\\[2cm]
%D. nombre tutor


\newpage\null\thispagestyle{empty}\newpage




\frontmatter

% Abstract en castellano
\renewcommand*\abstractname{Resumen}
\begin{abstract}
Las habilidades blandas o ``soft skills'' \cite{web:softSkills} son las habilidades personales y sociales que complementan las habilidades técnicas en el entorno laboral.

En los entornos de trabajo actuales, donde la mayoría de los trabajadores están altamente cualificados o sobrecualificados en algunos casos, las capacidades personales para interactuar con compañeros y responsables, así como para afrontar diversas tareas o situaciones de alto estrés, son cruciales para diferenciar entre un trabajador competente y uno ideal para un puesto de trabajo.

Analizar o contrastar datos sobre la evolución del estado de ánimo de los trabajadores o sobre el nivel de empatía entre personas o departamentos es una herramienta muy potente para las direcciones de recursos humanos y para las áreas de psicología de las empresas. Esto permite conocer de antemano las capacidades y debilidades de la plantilla, y anticiparse a situaciones predecibles.

El proyecto está dirigido a proporcionar a todas las empresas, tanto PYMES como grandes corporaciones, una herramienta accesible, fácilmente personalizable y que permite la gestión de un gran número de cuestionarios para analizar las habilidades necesarias.

RiskReal App es el resultado, y se encuentra disponible a través de la página del repositorio del proyecto: \url{https://github.com/Daniel-Fernandez-UBU/riskRealUBU}.

Su fácil instalación, simplicidad y escasa necesidad de recursos para funcionar son claves para su éxito.
\end{abstract}

\renewcommand*\abstractname{Descriptores}
\begin{abstract}
Spring boot, \textit{soft skills}, habilidades blandas, aplicación web, docker.
\end{abstract}

\clearpage

% Abstract en inglés
\renewcommand*\abstractname{Abstract}
\begin{abstract}
Soft skills \cite{web:softSkills} are personal and social abilities that complement technical skills in the workplace.

In today's work environments, where most workers are highly qualified or even overqualified in some cases, personal abilities to interact with colleagues and supervisors, as well as to handle various tasks or high-stress situations, are crucial to differentiate between a competent worker and an ideal candidate for a job.

Analyzing or comparing data on the evolution of employees' moods or the level of empathy between people or departments is a powerful tool for HR departments and company psychology teams. This allows them to understand in advance the strengths and weaknesses of the staff and to anticipate predictable situations.

The project aims to provide all companies, both SMEs and large corporations, with an accessible, easily customizable tool that allows the management of a large number of questionnaires to analyze the necessary skills.

RiskReal App is the result, and it is available through the project repository page: \url{https://github.com/Daniel-Fernandez-UBU/riskRealUBU}.

Its easy installation, simplicity, and low resource requirements are key to its success.
\end{abstract}

\renewcommand*\abstractname{Keywords}
\begin{abstract}
Spring boot, soft skills, web app, docker.
\end{abstract}

\clearpage

% Indices
\tableofcontents

\clearpage

\listoffigures

\clearpage

\listoftables
\clearpage

\mainmatter
\capitulo{1}{Introducción}

Las habilidades blandas, conocidas también como \textit{soft skills}, desempeñan un papel importante en la mayoría de las áreas de la vida, tanto en el ámbito profesional como en las relaciones interpersonales.

En el mundo laboral actual se están empezando a valorar más este tipo de habilidades, llegando a poder ser excluyentes. 

La capacidad de trabajar en equipo, una comunación efectiva o la ética del trabajo son solo algunos ejemplos de este tipo de habilidades, altamente demandadas por los responsables en los procesos de contratación, sobretodo en un entorno laboral donde la tipología de trabajo híbrido o 100\% presencial aparece en la mayoría de ofertas de empleo.

A nivel de contratación, la empresas no solo buscan a personas que estén cualificadas, sino que también se interesan por saber qué actitudes o cualidades tienen.

Poder encontrar un perfil que encaje con el entorno en el que va a trabajar, es crucial para el éxito de todas las partes, por eso no solo se deben evaluar este tipo de habilidades en nuevos procesos, sino que se tiene que realizar un proceso continuo de evaluación dentro de las empresas.

Conocer por qué en un grupo hay mucha rotación, o por qué en otros la mitad de los compañeros no se hablan entre sí es también fundamental para un crecimiento adecuado tando de la empresa como de las personas, un ambiente de trabajo agradable estimula la productividad y el espírutu de pertenencia que tanto se ha perdido últimamente.

Poder generar cuestionarios personalizados para estudiar una o varias habilidades de forma simúltanea es una gran ventaja para el análisis de conductas y actitudes.

RiskRealApp es una aplicación web, sencilla de desplegar y altamente personalizable, donde se asigna un valor a cada pregunta y donde los evaluadores pueden obtener un fichero de \textit{csv} con toda la información, separada por puntuación elegida a nivel de pregunta como a nivel total, por id de cuestionario o por fecha de realización, para poder generar tantas estadísticas como se necesiten.

El uso de tecnologías como \textit{docker} facilitan la implantación de esta aplicación, siendo portable, escalable y instalable en los principales sistemas operativos del mercado: \textit{Windows, Linux y macOS}.

\section{Estructura de la memoria}

La memoria tiene la siguiente estructura:

\begin{itemize}
	\item \textbf{Introducción}: Breve descripción de las habilidades blandas, y la importancia de poder medirla de forma continúa.
	\item \textbf{Objetivos del proyecto}: Exposición de los objetivos que persigue el proyecto.
	\item \textbf{Conceptos teóricos}: Explicación de los conceptos teóricos analizados para comprender la solución propuesta.
	\item \textbf{Técnicas y herramientas}: Técnicas y herramientas utilizadas en el desarrollo del proyecto.
	\item \textbf{Aspectos relevantes del desarrollo del proyecto}: Aspectos más destacables que han tenido lugar durante la realización del proyecto.
	\item \textbf{Trabajos relacionados}: Aplicación RiskReal\cite{web:riskreal} actual y páginas web similares.
	\item \textbf{Conclusiones y líneas de trabajo futuras}: Conclusiones obtenidas tras la finalización del proyecto y futuras mejoras que se podría aplicar.
\end{itemize}

De forma complementaria a la memoria, se proporcionan los siguientes anexos:

\begin{itemize}
	\item \textbf{Plan de proyecto software}: Planificación temporal y estudio de viabilidad del proyecto.
	\item \textbf{Especificación de requisitos}: Se definen los objetivos generales, los requisitos funcionales y no funcionales, y los diferentes casos de uso. 
	\item \textbf{Especificación de diseño}: Se describen las diferentes fases de diseño del proyecto.
	\item \textbf{Documentación técnica de programación}: Se describen los aspectos relacionados con el entorno de programación más relevantes, como la estructura de directorio o la forma de compilar una nueva versión de la aplicación.
	\item \textbf{Documentación de usuario}: Guía para una correcta instalación y manejo de la aplicación por parte de los usuarios.
	\item \textbf{Anexo de sostenibilidad curricular}: Aspectos relevantes de la sostenibilidad aplicados en el proyecto.
\end{itemize}

\section{Materiales adjuntos}

Los materiales que se adjuntan con la memoria son:

\begin{itemize}
	\item Aplicación java RiskRealApp.
	\item Cuestionarios de prueba.
	\item JavaDoc.
\end{itemize}

Los siguientes materiales están disponibles a través de internet:

\begin{itemize}
	\item Repositorio web del proyecto \cite{github:repo}.
	\item Imagen de docker de la aplicación \cite{github:dockerImage}.
\end{itemize}
\capitulo{2}{Objetivos del proyecto}

A continuación, se detalla los diferentes objetivos que han impulsado la realización del proyecto.

\section{Objetivos generales}

\begin{itemize}
	\item Desarrollar una aplicación web que permita la realización de cualquier tipo de cuestionario sobre habilidades blandas.
	\item Crear una aplicación sencilla y portable, fácilmente utilizable, y que permita cierta configuración adicional de base.
	\item Almacenar los resultados en un archivo estructurado y sencillo de analizar.
	\item Permitir que la aplicación pueda llegar a la mayoría de los países del entorno, estando disponible en varios idiomas.
\end{itemize}



\section{Objetivos técnicos}


\begin{itemize}
	\item Utilizar un \textit{framework} de java para el desarrollo de la aplicación.
	\item Convertir la aplicación web en lo más portable posible, utilizando tecnologías como \textit{docker}.
	\item Aplicar la arquitectura \textit{modelo-vista-controlador}. separando la parte del código que se encarga de presentar la imagen de la que se encarga de obtener los datos.
	\item Incluir la conexión a una base de datos para almacenar los datos de los usuarios de forma persistente.
	\item Utilizar GitHub como plataforma de control de versiones.
	\item Utilizar \textit{maven} como herramienta de automatización en la construcción de la aplicación y la imagen de \textit{docker}.
\end{itemize}

\section{Objetivos personales}

\begin{itemize}
	\item Adentrarme en el \textit{framework de Spring Boot}. algo totalmente desconocido para mí hasta el momento, y permitirme ver lo sencillo que es la creación de una aplicación web, una vez que se conocen todas sus bondades.
	\item Explorar el ecosistema completo de la programación web, tanto por lo anterior como por el uso de motores de plantillas y de generación de código \textit{HTML} de forma dinámica con \textit{Thymeleaf}.
	\item Profundizar en \textit{docker} y su ecosistema, una tecnología con gran auge en el mercado laboral actual.
	\item La creación de una aplicación que tenga claramente diferenciado el \textit{front-end} (páginas \textit{HTML}) del \textit{back-end} (código en java) y con persistencia fuera de la aplicación (base de datos y almacenamiento local o centralizado en el equipo anfitrión).
\end{itemize}
\capitulo{3}{Conceptos teóricos}

La parte del proyecto que me ha supuesto mayor complejidad teórica ha sido la relacionada con la tecnología de \textit{Spring Boot} ~\cite{web:springboot}.
Entender cómo funciona \textit{docker} ~\cite{web:dockerDocs} tampoco ha sido una tarea sencilla, pero el sí necesaria, tanto para hacer más accesible la aplicación como para potenciar mis habilidades técnicas.

\section{Spring Boot}

Spring Boot ~\cite{web:springbootArch} es la evolución del clásico \textit{framework} de creación de aplicaciones web.

La construcción de una aplicación web con un \textit{framework} antiguo consistía en los siguientes pasos:

\begin{enumerate}
	\item Crear un proyecto de java.
	\item Importar las dependencias necesarias.
	\item Preparar todos los archivos de configuración necesarios para que funcione la aplicación.
	\item Desplegar la aplicación en un servidor web.
\end{enumerate}

Este mismo proceso con Spring Boot consta de los siguientes pasos:

\begin{enumerate}
	\item Acceder a \textit{Spring Initializr} ~\cite{web:springinitializr}.
	\item Indicar el tipo de proyecto, el lenguaje de programación y los \textit{starters} \cite{web:springStarters} de Spring necesarios.
	\item Generar el proyecto.
\end{enumerate}

El paso 2 de la lista anterior engloba todos los pasos de la primera lista, incluso va más allá, porque el proyecto que genera se puede arrancar desde la propia \textit{suite} de Spring y tendrías tu aplicación web funcionando en muy poco tiempo.

\subsection{\textit{Starters} Utilizados}

A continuación, se van a comentar ciertos aspectos de los \textit{starters} que se han utilizado en la construcción de la aplicación.

\subsubsection{Spring Web}

Establece la configuraión del proyecto para su funcionamiento como una aplicación web. Al incluir un servidor \textit{Apache Tomcat} no es necesario generar el archivo \textit{JAR o WAR} y desplegarlo en un servidor, solo con levantar la aplicación esta ya se lanza en el servidor web embebido.
 

\subsubsection{Spring Security}

Establece la configuración de seguridad por defecto en la aplicación web. Sin más configuración adicional, genera una contraseña por defecto de administrador con cada inicio de la aplicación, ya que securiza todas las páginas web de forma predeterminada.

Con ciertas clases de java, Spring Security ~\cite{web:springSecurity} permite personalizar la configuración de la aplicación.

\subsubsection{Spring JPA}

Establece toda la integración para usar clases relacionadas con tablas de una base de datos. Elimina la necesidad de crear patrones DAO ~\cite{web:dao} para nuestras clases y simplifica este proceso de forma considerable.

\begin{enumerate}
	\item Creamos nuestra clase de java.
	\item Incluimos las anotaciones ~\cite{web:anotacionesJPA} en la clase.
	\item Creamos una interfaz que extienda de \textit{JPA repository}  ~\cite{web:springCJPArepo} o de \textit{CRUD repository} ~\cite{web:springCRUDrepo} asociada a nuestra clase.
	\item Ya disponemos de todo lo necesario para actualizar, borrar, crear o recuperar objetos de la base de datos.
\end{enumerate}

\subsubsection{Thymeleaf}

Es un motor de generación de plantillas y de generación dinámica de código \textit{HTML}.

Permite incluir cierta lógica en las etiquetas \textit{HTML} de nuestras páginas web para mostrar u ocultar opciones, iterar sobre un objeto o lista para crear tantas filas en una tabla como valores.

Permite el poder enviar variables desde nuestro controlador de java a la página web, reconocer esas variables y mostrar su contenido.

También incluye el soporte para los archivos de personalización, pudiendo acceder a los archivos de propiedades en función del idioma seleccionado en la web.

\subsubsection{Docker Compose Support}

Permite tener un fichero de \textit{docker compose} dentro del propio proyecto de java, para levantar el contenedor que sea necesario en el momento de inicio de la aplicación.

Por ejemplo, si se utiliza una base de datos, se puede configurar que esta se levante en un contenedor a la vez que se ejecuta el proyecto, para no tener que configurar una base de datos fuera del mismo, simplificando el entorno de desarollo.

\section{Docker}

Un \textit{docker} o contenedor, es una imagen de un sistema operativo básico con una o varias aplicaciones instaladas que al arrancarse, permite el acceder a esta aplicación sin necesidad de dedicar una gran fuente de recursos a su funcionamiento.

Es realmente mucho más sencillo que todo eso, se podría simplificar con lo siguiente: \textbf{Docker es un servidor corriendo como un microservicio}.

Lo mejor de Docker es que las imágenes son multiplataforma, con tener un servicio de docker instalado en el equipo, puede correr cualquiera de las imágenes disponible en su repositorio.

Facilita el despliegue y portabilidad de nuevas aplicaciones, a la vez que disminuye el tamaño de la infraestructura necesaria para su funcionamiento.




\capitulo{4}{Técnicas y herramientas}


En esta parte de la memoria se van a presentar las distintas herramientas y técnicas utilizadas en el desarollo del proyecto.
Se incluirán todas las que se hayan valorado o utilizado, incluso aquellas que tras ciertas pruebas o pasos, se hayan descartado por haber encontrado una alternativa más funcional.
A continuación, una tabla resumen con las herramientas utilizadas en cada parte del proyecto:

\tablaSmall{Herramientas y tecnologías utilizadas en cada parte del proyecto}{l c c c c}{herramientasportipodeuso}
{ \multicolumn{1}{l}{Herramientas} & App Web & BBDD & Memoria \\}{ 
HTML5 & X & & \\
CSS & X & &\\
BOOTSTRAP & X & &\\
Java & X & &\\
Spring Boot & X & &\\
Eclipse & X & &\\
Thymeleaf & X & &\\
Docker & X & X &\\
JSON & X & X &\\
CSV & X & X &\\
MySQL & & X & X\\
GitHub & X & X & X\\
Zube.io & X & X & X\\
Mik\TeX{} & & & X\\
\LaTeX  & & & X\\
} 

\section{Spring Tool Suite for Eclipse}

Herramienta de desarrollo (IDE) que se utilizará para llevar a cambo el proyecto.
A pesar de haber otras similares, como InteliJ, he preferido utilizar la \textit{suite} de \textit{Spring Boot} para Eclipse al estar más familiarizado con el entorno de trabajo, ya que ha sido la herramienta utilizada en otras asignaturas del grado.

\section{\LaTeX}
El editor de texto para generar toda la documentación relacionada con el trabajo de fin de grado será \LaTeX.
Aún estándo acostumbrado a utilizar Microsoft Word para este tipo de tareas, veo en utilizar \LaTeX  una oportunidad de aprender un nuevo generador de documentos.
La curva de aprendizaje es más grande que con Word (en mi caso), pero el resultado merece la pena porque permite aplicar formatos de forma sencilla, un aspecto en el que Word para documentos muy grandes, está en clara desventaja.

\section{Mik\TeX{}}
Para trabajar con plantillas de \LaTeX  en entornos windows existen varias alternativas, en mi caso he preferido utilizar el programa Mik\TeX{}.
Es un programa sencillo y tiene por separado tanto la parte de edición como la de presentación, para ir viendo los cambios una vez se compila la plantilla completa.

\section{Spring Boot}

Spring Boot ~\cite{web:springbootArch} es la evolución del clásico \textit{framework} de creación de aplicaciones web.

La construcción de una aplicación web con un \textit{framework} antiguo consistía en los siguientes pasos:

\begin{enumerate}
	\item Crear un proyecto de java.
	\item Importar las dependencias necesarias.
	\item Preparar todos los archivos de configuración necesarios para que funcione la aplicación.
	\item Desplegar la aplicación en un servidor web.
\end{enumerate}

Este mismo proceso con Spring Boot consta de los siguientes pasos:

\begin{enumerate}
	\item Acceder a \textit{Spring Initializr} ~\cite{web:springinitializr}.
	\item Indicar el tipo de proyecto, el lenguaje de programación y los \textit{starters} \cite{web:springStarters} de Spring necesarios.
	\item Generar el proyecto.
\end{enumerate}

El paso 2 de la lista anterior engloba todos los pasos de la primera lista, incluso va más allá, porque el proyecto que genera se puede arrancar desde la propia \textit{suite} de Spring y tendrías tu aplicación web funcionando en muy poco tiempo.

\subsection{\textit{Starters} Utilizados}

A continuación, se van a comentar ciertos aspectos de los \textit{starters} que se han utilizado en la construcción de la aplicación.

\subsubsection{Spring Web}

Establece la configuraión del proyecto para su funcionamiento como una aplicación web. Al incluir un servidor \textit{Apache Tomcat} no es necesario generar el archivo \textit{JAR o WAR} y desplegarlo en un servidor, solo con levantar la aplicación esta ya se lanza en el servidor web embebido.
 

\subsubsection{Spring Security}

Establece la configuración de seguridad por defecto en la aplicación web. Sin más configuración adicional, genera una contraseña por defecto de administrador con cada inicio de la aplicación, ya que securiza todas las páginas web de forma predeterminada.

Con ciertas clases de java, Spring Security ~\cite{web:springSecurity} permite personalizar la configuración de la aplicación.

\subsubsection{Spring JPA}

Establece toda la integración para usar clases relacionadas con tablas de una base de datos. Elimina la necesidad de crear patrones DAO ~\cite{web:dao} para nuestras clases y simplifica este proceso de forma considerable.

\begin{enumerate}
	\item Creamos nuestra clase de java.
	\item Incluimos las anotaciones ~\cite{web:anotacionesJPA} en la clase.
	\item Creamos una interfaz que extienda de \textit{JPA repository}  ~\cite{web:springJPArepo} o de \textit{CRUD repository} ~\cite{web:springCRUDrepo} asociada a nuestra clase.
	\item Ya disponemos de todo lo necesario para actualizar, borrar, crear o recuperar objetos de la base de datos.
\end{enumerate}

\section{Thymeleaf}

Es un motor de generación de plantillas y de generación dinámica de código \textit{HTML}.

Permite incluir cierta lógica en las etiquetas \textit{HTML} de nuestras páginas web para mostrar u ocultar opciones, iterar sobre un objeto o lista para crear tantas filas en una tabla como valores.

Permite el poder enviar variables desde nuestro controlador de Java a la página web, reconocer esas variables y mostrar su contenido.

También incluye el soporte para los archivos de personalización, pudiendo acceder a los archivos de propiedades en función del idioma seleccionado en la web.

El uso de los \textit{fragments} es un gran aliado para ahorrarnos líneas de código.

\section{HTML}

Lenguaje utilizado en la creación de las páginas web, en particular HTML5, que tiene algunas funcionalidades básicas ya implementadas, como por ejemplo el uso de ``required'' en campos de formulario para que sea obligatorio rellenar los campos sin necesidad de utilizar JavaScript para el control.

\section{MySQL}

Se utiliza MySQL como base de datos para almacenar los usuarios y roles.
Se ha decidido usar una base de datos porque su integración con Spring Security es muy sencilla y funcional. 
A través de los repositorios CRUD que nos proporciona Spring JPA podemos gestionar las tablas y registros de forma ágil.

\section{Docker}

Un \textit{docker} o contenedor, es una imagen de un sistema operativo básico con una o varias aplicaciones instaladas que al arrancarse, permite el acceder a esta aplicación sin necesidad de dedicar una gran fuente de recursos a su funcionamiento.

Es realmente mucho más sencillo que todo eso, se podría simplificar con lo siguiente: \textbf{Docker es un servidor corriendo como un microservicio}.

Lo mejor de Docker es que las imágenes son multiplataforma, con tener un servicio de docker instalado en el equipo, puede correr cualquiera de las imágenes disponible en su repositorio.

Facilita el despliegue y portabilidad de nuevas aplicaciones, a la vez que disminuye el tamaño de la infraestructura necesaria para su funcionamiento.

Se ha utilizado docker tanto para el servidor de MySQL como para la propia aplicación Web, de esta forma con 2 simples contenedores, un docker-compose (fichero de configuración de dockers que permite descargarse, crear, parar o lanzar dockers) podemos tener toda la aplicación completa corriendo en cualquier entorno, ya que docker funciona en los principales sistemas operativos del mercado: Linux, Windows y macOS.

\subsection{Docker Compose}

Esta herramienta complementaria de docker, \textit{docker compose} \cite{web:dockercompose} permite definir y gestionar aplicaciones que requieran más de un contenedor, de forma conjunta y sencilla.
Se ha utilizado como base para la instalación del proyecto, explicando sus funcionalidades y permitiendo ciertas personalizaciones en el despligue de la aplicación.

\section{GitHub}

Se va utiliza GitHub para dar visibilidad al trabajo diario y constante en el proyecto, ya que permite realizar aportaciones incrementales de código, documentación, etc del TFG completo.

Es la herramienta por excelencia para llevar un control de versiones en cualquier proyecto, sea software o no.

\subsection{Repository}

Se ha creado un repositorio~\cite{github:repo} donde se subirá todo el TFG al completo, tanto la documentación de la memoria, como la aplicación de java.

\subsection{Releases}

En esta sección dentro del repositorio se han ido subiendo las diferentes versiones funcionales de la aplicación, para tener un lugar central desde el que descargarse la última versión ``liberada''.

\subsection{Packages}

En esta parte de GitHub se han subido las imágenes de docker funcionales de la aplicación, para pemitir un despliegue sencillo desde una ubicación centralizada.
Con un simple comando de \textit{docker pull url de la imagen} puede descargarte la imagen para empezar a utilizarla.

\section{JSON}

Se van a utilizar ficheros con la estructura de un json para almacenar los cuestionarios completos, a los que se accederá desde la aplicación web de Java.

\section{Zube.io}

Se va a utilizar la versión gratuita de Zube.io, que permite generar Sprints, y gráficos de seguimiento, como \textit{Burndown} o \textit{Burnup}.

Gracias a la sencilla y completa integración con Github ofrece una versión más completa de la parte de Proyect de Github, por lo que he decidido utilizar esta plataforma para el seguimiento de las tareas en vez de la que ofrecía Github.

La versión gratuita permite integrar lo necesario del repositorio y del proyecto, por lo que no es necesario ampliar a versiones más completas.

Aunque se puede compartir con otros usuarios y colabores del proyecto, no se permite el acceso público.

\section{GitPod}

Se ha estudiado la posibilidad de usar \textit{gitpod} para alojar una versión funcional de la aplicación, ya que diferentes alternativas como \textit{Heroku} o \textit{Render} ni permitían un fácil despliegue con docker-compose.yml ni daban la opción de una prueba real de forma gratuita.

A pesar de haber vinculado \textit{gitpod} con \textit{GitHub}, se limitaba a 50 horas el despliegue de una aplicación, pero eso no se incluye como parte del proyecto.









\capitulo{5}{Aspectos relevantes del desarrollo del proyecto}

Este apartado pretende recoger los aspectos más interesantes del desarrollo del proyecto, incluyendo las diferentes decisiones que se han ido tomando y los problemas que se han tenido que afrontar.

\section{Inicio del proyecto}

La selección del proyecto suponía un doble reto personal:

\begin{itemize}
	\item Nunca había programado aplicaciones web ni conocía la tecnología de Spring Boot.
	\item No había un proyecto base, documentado, como inicio del proyecto, sino que era crear algo nuevo desde cero.
\end{itemize}

Tras la primera reunión con el tutor, donde me expuso el camino que se quería seguir con el proyecto, me puse a recabar información sobre é.

\section{Metodologías}

Desde el inicio del proyecto se ha dedicado tiempo y esfuerzo en que la realización del proyecto fuese de la manera más profesional posible.

La gestión del proyecto se ha realizado teniendo como base la metodología ágil de \textit{Scrum} ~\cite{web:scrum}.

Al tratarse de un Trabajo de Fin de Grado, que se realiza de forma individual, no se siguió de forma completa, porque el grupo de desarrolladores estaba formado solo por un miembro.

Cómo se ha aplicado la metodología:

\begin{itemize}
	\item Reunión inicial de \textit{kick-off} ~\cite{web:kickoff} del proyecto, donde se presentó el proyecto y se establecieron ciertos objetivos generales.
	\item Se utilizó una estrategia de desarrollo incremental basada en \textit{sprints}, a través de iteraciones y revisiones.
	\item La duración media de los \textit{sprints} fue de dos semanas, excepto los últimos del proyecto, que han sido de una semana.
	\item En la finalización de cada \textit{sprints} se entregaba un incremento del proyecto.
	\item Se realizaba una reunión de revisión del \textit{sprint} que servía también como reunión de planificación del nuevo \textit{sprint}.
	\item Se definía en el tablero kanban de \textit{Zube.io} una lista de tareas y se asociaban al nuevo \textit{sprint}.
	\item Para monitorizar el progreso se han utilizado gráficos \textit{burndup}.
\end{itemize}

\section{Formación}

La principal tecnología en el desarrollo del proyecto ha sido \textit{Spring Boot}, sobre la que no se tenía ningún conocimiento previo, lo que me ha llevado a tener que formarme por diferentes medios:

\begin{itemize}
	\item Spring Boot 3. Aplicaciones web y REST APIs con Spring MVC (Udemy) ~\cite{udemy:eliseo}.
	\item Cómo convertir en imagen de docker una aplicación de Spring Boot ~\cite{doc:springbootdocker}.
	\item Cómo aplicar estilos de personalización de \textit{bootstrap} y \textit{CSS} en documentos \textit{HTML}.
	\item Uso y configuración de un archivo \textit{docker compose} ~\cite{web:dockercompose}.
	\item Cómo se utiliza \LaTeX para la generación de documentos.
\end{itemize}

Me gustaría destacar la web \href{https://www.baeldung.com/spring-boot}{Baeldung}, al contener mucha información sobre Spring Boot en general.

\section{Desarrollo de la aplicación web}

Se va a dividir el desarrollo de la aplicación web en varios apartados, para poder detallar mejor el proceso de construcción de la misma.

\subsection{Código de la aplicación}

El desarrollo del código de la aplicación supuso un primer reto, pues en el inicio del proyecto todavía no terminaba de comprender cómo iba a crear una aplicación web con java, ya que no era lo que estaba acostumbrado a crear con este lenguaje de programación.

Una vez se comprendió cómo funcionaba, de forma básica, \textit{Spring Web} como \textit{starter} del proyecto, este paso se convirtió en algo trivial.

La primera parte de la creación de la aplicación consistía en poder ``leer'' un archivo \textit{JSON} y convertirlo a objetos en java con los que se pudiese trabajar de forma nativa.

Se crearon varias clases modelo para agrupar el contenido del \textit{JSON}:

 \begin{itemize}
	\item Clase \textit{Quiz}: Contiene la descripción básica del cuestionario y un listado de preguntas.
	\item Clase \textit{Questions}: Contiene la información de cada pregunta y un listado de respuestas.
	\item Clase \textit{Answers}: Contiene la información de cada respuesta.
\end{itemize}

Una vez conseguido lo anterior, la base del proyecto ya estaba construida.

Para la presentación de contenidos en las páginas web, se usaron clases \textit{controller} en java, que sirven de enlace entre la propia aplicación y el código \textit{HTML} de la página web.

A través de estas clases, se pasaban los cuestionarios completos al código \textit{HTML} de la página web y en ella, con \textit{Thymeleaf} se mostraba la información.

Ya se tenía la forma de ver el contenido del \textit{JSON} en las páginas web.

Hasta aquí tendríamos una base más ``robusta'' de la aplicación, teniendo todo lo necesario dentro del propio proyecto de java.

Los primeros problemas los encontramos al intentar actualizar los archivos de propiedades, ya que al ser internos de la aplicación, necesitaban un reinicio de la misma cada vez que había un cambio, algo que no era funcional para un entorno de producción.

A raíz de lo anterior, se empezó a trabajar con archivos \textit{JSON} externos a la aplicación, utilizando rutas relativas para poder encontrar los archivos siempre, independientemene de la ubicación raíz del proyecto.

Finalmente, se incluyeron varios tipos de clases de java adicionales:

 \begin{itemize}
	\item Clases de configuración: Se utilizaron ciertas clases para modificar la configuración por defecto de Spring Boot. También se ha utilizado para acceder a un archivo de propiedades personalizado, donde se definen algunos elementos de configuración y personalización de la aplicación.
	\item Clases de servicio: Contienen los métodos a los que se tienen que acceder desde otras clases, evitando duplicidad en el código y teniendo una clase centralizada para la gestión de la mayoría de los procesos, siendo más sencillo comprender su código y ahorrando métodos adicionales en otras clases.
\end{itemize}

\subsection{Código de las páginas web}

La estética y el diseño de las páginas web ha sido lo más complejo en cuanto a presentación final del proyecto.

Ayudado de \textit{bootstrap} y personalización a través de plantillas \textit{CSS} se ha conseguido aplicar un estilo homogéneo a toda la aplicación.

El uso de los fragmentos de \textit{Thymeleaf} ha ayudado a que los ficheros \textit{HTML} que componen la página web sean más legibles, evitando tener que escribir las mismas líneas de forma repetitiva, simplificándolo todo a una simple línea que enlaza con el \textit{fragmento} indicado.

Definitivamente, implementar las etiquetas \textit{HTML} ha sido lo más costoso en cuanto a programación se refiere, porque conseguir cuadrar y encajar todos los apartados en su sitio ha sido un trabajo de prueba y error constante.

\subsection{Integración de servicios y dependencias}

Los servicios o dependencias más destacables que implementa la aplicación son los siguientes:

\subsubsection{Servicio de envío de correos electrónicos}

Durante el proceso de desarrollo de un servicio de correo electrónico que enviase a los usuarios una nueva contraseña en caso de que no recordasen la suya, se probaron configuraciones con varios servidores de correo, pero finalmente, por simplifidad y acceso a documentación, se utilizó \textit{Gmail} que permite la creación de cuentas gratuitas y el uso de sus servidores de correo desde cualquier \textit{API} siempre que se utilice un \textit{token} ~\cite{gmail:token}.

La configuración completa del servidor de correo, así como, sus credenciales, se independizaron de la aplicación en un archivo de propiedades, para permitir su modificación por cualquier otro servicio de correo, dando libertar al programador para cambiarlo sin necesidad de compilar de nuevo el proyecto.

\subsubsection{Seguridad con Spring Security y Spring JPA}

En primer lugar, tras aplicar \textit{Spring Security}, la mayoría de las páginas dejaron de funcionar, todas aquellas que eran de tipo \textit{POST}.

Tras investigar descubrí que esto sucedía porque la primera medida de seguridad que se aplica, es la necesidad de recibir un \textit{CSRF token} (\textit{Cross-Site Request Forgery}) ~\cite{web:csrf} para evitar ataques \textit{man in the middle} a los formularios de las aplicaciones web, al requerir un \textit{token} de autenticación, para comprobar la autenticidad de la comunicación.

Debido a esto, todos los formularios tienen configurado un campo ``oculto'' que envía este \textit{token} al controlador de la aplicación, para permitir el acceso.

Se decidió utilizar una base de datos \textit{mysql} que se levanta cada vez que se inicia la aplicación, almacena los datos en una carpeta externa a la misma, para tener persistencia, y que está vinculada con el la seguridad de la aplicación, ya que se definieron los campos de nuestras tablas de usuarios y perfiles que se iban a usar tanto para iniciar sesión, como para comprobar a qué parte de la aplicación tiene cada uno.

\section{Documentación}

Para generar la documentación del proyecto, se ha utilizado una plantilla de \LaTeX, que permite mantener una estructura uniforme y el mismo estilo durante la documentación de todo el proyecto, aportando un aura de profesionalidad y sin los poblemas y complicaciones de formatos que nos podemos encontrar en otros programas comerciales.

La documentación de la aplicación se ha generado desde el propio editor de código, \textit{Spring Tool Suite 4 for Eclipse}, siguiendo el formato \textit{javadoc} ~\cite{doc:javadoc}.

Este formato permite la visualización de todas las clases utilizadas en formato \textit{HTML}, pudiendo navegar por los diferentes métodos y obteniendo una breve descripción de cada uno, así como de los parmáetros de entrada que esperan o del retorno de cada método.
\clearpage
\section{Publicación}

Se ha intentado realizar la publicación de la aplicación en la plataforma en la nube de \textit{GitPod} ~\cite{gitpod}, porque permitía la integración de un repositorio de \textit{GitHub} y poder desplegar tus aplicaciones con un archivo \textit{docker-compose.yml}, pero las limitaciones del plan gratuito en cuanto a horas mensuales de ejecución no han permitido tenerla disponible de forma continua, si puediendo comprobarse su correcto despliegue y funcionamiento.

Despliegue de la aplicación con ``docker compose up -d'' en \textit{GitPod}:
\imagen{gitPodDespliegue}{Despliegue de la aplicación en la plataforma web de GitPod.}{.5}

Acceso a la aplicación web a través de la dirección web que genera \textit{GitPod}:
\imagen{gitPodWeb}{Funcionamiento de la aplicación web a través de GitPod.}{.5}





\capitulo{6}{Trabajos relacionados}

Este apartado sería parecido a un estado del arte de una tesis o tesina. En un trabajo final grado no parece obligada su presencia, aunque se puede dejar a juicio del tutor el incluir un pequeño resumen comentado de los trabajos y proyectos ya realizados en el campo del proyecto en curso. 

\section{Web RiskReal.eu}
Es la página web \cite{web:riskreal} que se ha utilizado como base para la creación del nuevo proyecto.
La aplicación se basa en la utilización de diferentes cuestionarios para evaluar "soft skills" de trabajadores.

\subsection{Descripción general}
Desde la página principal se ofrece una descripción básica de lo que se puede hacer.
Indica que con diversos escenarios de test, se pueden evaluar de forma eficiente las diferentes "soft skills".

Consta de dos páginas base, "Inicio", que consideraría la parte "abierta" de la web, y "Cursos", con acceso restringido solo para usuarios registrados.

\subsection{Parte abierta}
Permite realizar un cuestionario propio, para obtener una evaluación orientativa de en qué estado nos encontramos en cuanto a habilidades blandas; y un test sobre un posible escenario, donde el "trabajador" va contestando en función de diferentes situaciones.

\subsection{Parte privada}
Aunque no tengo acceso a esta parte, es la que permite a las empresas generar sus propios cuestionarios para evaluar a sus trabajadores.
Permite esa personalización necesaria para que en función del sector o lo que se quiera evaluar, se pueda generar algo específico.




\capitulo{7}{Conclusiones y Líneas de trabajo futuras}

En este capítulo se exponen las conclusiones derivadas del trabajo del proyecto, así como las líneas de trabajo futuras por las que se puede dar continuidad al proyecto.

\section{Conclusiones}

Tras el desarrollo del proyecto se obtienen las siguientes conclusiones:

\begin{itemize}
	\item Se ha cumplido con el objetivo general del proyecto, teniendo ahora disponible una aplicación web que permite la realización y gestión de cuestionarios.
	\item Haber utilizado tecnologías como Spring Boot y Docker, le dan al proyecto un aura de modernidad, al ser tecnologías que están en expansión dentro del ámbito del desarrollo de aplicaciones.
	\item He podido aplicar conocimientos de gran parte de lo aprendido durante el grado, y aprender sobre muchos otros en los que no contaba con ninguna experiencia previa.
	\item Integrar la creación del proyecto con herramientas de control de versiones ha facilitado el desarrollo incremental del proyecto, así como un \textit{know-how} de buenas prácticas para cualquier proyecto \textit{software} que tenga que desarrollar en el futuro.
	\item Estimar el tiempo total dedicado a pruebas y prototipos de la aplicación es muy complicado, ya que en algunos momentos me he atascado con parte de la programación y he tenido que probar vías alternativas, pero el uso de la metodología ágil para la gestión de las tareas ha facilitado el seguimiento del progreso.
\end{itemize}

\section{Líneas de trabajo futuras}

La entrega del Trabajo de Fin de Grado solo es el inicio del proyecto, ya que su desarrollo contiúa.

A continuación, se resume el camino a seguir del proyecto:

\begin{itemize}
	\item Integrar en la página web la posibilidad de editar los cuestionarios ya cargados, sin tener que actualizar el archivo \textit{JSON} y volver a cargarlo.
	\item Integrar la aplicación con una \textit{API} de traducción que permita generar de forma automática los cuestionarios en otros idiomas.
	\item Integrar la posibilidad de crear cuestionarios desde cero desde la propia página web, contando con la ventaja de tener un modelo de datos ya definido.
	\item Se puede considerar la opción de migrar la aplicación a plataformas móviles, para no depender de la conexión a internet para acceder.
\end{itemize}





\bibliographystyle{plain}
\bibliography{bibliografia}

\end{document}
