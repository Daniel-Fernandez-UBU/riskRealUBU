\capitulo{1}{Introducción}

Las habilidades blandas, conocidas también como \textit{soft skills}, desempeñan un papel importante en la mayoría de las áreas de la vida, tanto en el ámbito profesional como en las relaciones interpersonales.

En el mundo laboral actual se están empezando a valorar más este tipo de habilidades, llegando a poder ser excluyentes. 

La capacidad de trabajar en equipo, una comunación efectiva o la ética del trabajo son solo algunos ejemplos de este tipo de habilidades, altamente demandadas por los responsables en los procesos de contratación, sobretodo en un entorno laboral donde la tipología de trabajo híbrido o 100\% en modalidad de teletrabajo aparece en la mayoría de ofertas de empleo.

A nivel de contratación, la empresas no solo buscan a personas que estén cualificadas, sino que también se interesan por saber qué actitudes o cualidades tienen.

Poder encontrar un perfil que encaje con el entorno en el que va a trabajar es crucial para el éxito de todas las partes, por eso no solo se deben evaluar este tipo de habilidades en nuevos procesos, sino que se tiene que realizar un proceso continuo de evaluación dentro de las empresas.

Conocer por qué en un grupo hay mucha rotación, o por qué en otros la mitad de los compañeros no se hablan entre sí es también fundamental para un crecimiento adecuado tando de la empresa como de las personas, porque un ambiente de trabajo agradable estimula la productividad y el espíritu de pertenencia que tanto se ha perdido últimamente.

Poder generar cuestionarios personalizados para estudiar una o varias habilidades de forma simúltanea es una gran ventaja para el análisis de conductas y actitudes.

RiskRealApp es una aplicación web, sencilla de desplegar y altamente personalizable, donde se asigna un valor a cada pregunta y donde los evaluadores pueden obtener un fichero de \textit{CSV} con toda la información, separada por puntuación elegida tanto a nivel de pregunta como a nivel total, por \textit{ID} de cuestionario o por fecha de realización, para poder generar tantas estadísticas como se necesiten.

El uso de tecnologías como \textit{docker} facilitan el despliegue de esta aplicación, siendo portable, escalable e instalable en los principales sistemas operativos del mercado: \textit{Windows, Linux y macOS}.

\section{Estructura de la memoria}

La memoria tiene la siguiente estructura:

\begin{itemize}
	\item \textbf{Introducción}: Breve descripción de las habilidades blandas, y la importancia de poder medirlas de forma continua.
	\item \textbf{Objetivos del proyecto}: Exposición de los objetivos que persigue el proyecto.
	\item \textbf{Conceptos teóricos}: Explicación de los conceptos teóricos analizados para comprender la solución propuesta.
	\item \textbf{Técnicas y herramientas}: Técnicas y herramientas utilizadas en el desarrollo del proyecto.
	\item \textbf{Aspectos relevantes del desarrollo del proyecto}: Aspectos más destacables que han tenido lugar durante la realización del proyecto.
	\item \textbf{Trabajos relacionados}: Aplicación RiskReal\cite{web:riskreal} actual y páginas web similares.
	\item \textbf{Conclusiones y líneas de trabajo futuras}: Conclusiones obtenidas tras la finalización del proyecto y futuras mejoras que se podría aplicar.
\end{itemize}

De forma complementaria a la memoria, se proporcionan los siguientes anexos:

\begin{itemize}
	\item \textbf{Plan de proyecto software}: Planificación temporal y estudio de viabilidad del proyecto.
	\item \textbf{Especificación de requisitos}: Se definen los objetivos generales, los requisitos funcionales y no funcionales, y los diferentes casos de uso. 
	\item \textbf{Especificación de diseño}: Se describen las diferentes fases de diseño del proyecto.
	\item \textbf{Documentación técnica de programación}: Se describen los aspectos relacionados con el entorno de programación más relevantes, como la estructura de directorios o la forma de compilar una nueva versión de la aplicación.
	\item \textbf{Documentación de usuario}: Guía para una correcta instalación y manejo de la aplicación por parte de los usuarios.
	\item \textbf{Anexo de sostenibilidad curricular}: Aspectos relevantes de la sostenibilidad aplicados en el proyecto.
\end{itemize}

\section{Materiales adjuntos}

Los materiales que se adjuntan con la memoria son:

\begin{itemize}
	\item Aplicación java RiskRealApp.
	\item Cuestionarios de prueba.
	\item JavaDoc.
\end{itemize}

Los siguientes materiales están disponibles a través de internet:

\begin{itemize}
	\item Repositorio web del proyecto \cite{github:repo}.
	\item Imagen de docker de la aplicación \cite{github:dockerImage}.
\end{itemize}