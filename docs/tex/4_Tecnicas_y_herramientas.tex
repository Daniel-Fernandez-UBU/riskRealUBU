\capitulo{4}{Técnicas y herramientas}


En esta parte de la memoria se van a presentar las distintas herramientas y técnicas utilizadas en el desarollo del proyecto.
Se incluirán todas las que se hayan valorado o utilizado, incluso aquellas que tras ciertas pruebas o pasos, se hayan descartado por haber encontrado una alternativa más funcional.

\section{Spring Tool Suite for Eclipse}

Herramienta de desarrollo (IDE) que se utilizará para llevar a cambo el proyecto.
A pesar de haber otras similares, como InteliJ, he preferido utilizar la \textit{suite} de \textit{Spring Boot} para Eclipse al estar más fmailiarizado con el entorno de trabajo, ya que ha sido la herramienta utilizada en otras asignaturas del grado.

\section{\LaTeX}
El editor de texto para generar toda la documentación relacionada con el trabajo de fin de grado será \LaTeX.
Aún estándo acostumbrado a utilizar Microsoft Word para este tipo de tareas, veo en utilizar \LaTeX  una oportunidad de aprender un nuevo generador de documentos.
La curva de aprendizaje es más grande que con Word (en mi caso), pero el resultado merece la pena porque permite aplicar formatos de forma sencilla, un aspecto en el que Word para documentos muy grandes, está en clara desventaja.

\section{Mik\TeX{}}
Para trabajar con plantillas de \LaTeX  en entornos windows existen varias alternativas, en mi caso he preferido utilizar el programa Mik\TeX{}.
Es un programa sencillo y tiene por separado tanto la parte de edición como la de presentación, para ir viendo los cambios una vez se compila la plantilla completa.

\section{Spring Boot}

Para la aplicación web en java se utilizará el Framework de Spring.
Es mi primera vez con un framework en java, y para aprender conceptos de forma más rápida y con ejemplos he tomado como referencia un curso en Udemy \cite{udemy:eliseo} para dar un acelerón en el aprendizaje.
Gracias a las dependencias predefinidas, facilita el desarrollo de la aplicación de forma considerable.

\subsection{Spring Web}

Es la dependencia principal de Spring Boot para la creación de aplicaciones web. 
Simplifica la creación de controladores, el manejo de las solicitude y se integra de forma sencilla con motores de plantillas como Thymeleaf.

\subsection{Spring JPA}

Es la dependencia de Spring para el manejo de entidades persistentes de una base de datos. 
Con los respositorios CRUD o JPA, y la inclusión por defecto de Hibernate, simplifica hasta el extremo la gestión de nuestros los objetos en la base de datos.

\subsection{Spring Security}

Es la dependencia de Spring para dotar de seguridad nuestra aplicación de forma automática.

Esta seguridad se basa en autenticación y autorizació principalmente, para poder permitir ciertas partes de nuestra web a usuarios "invitados" o limitar el acceso a otras en función del rol del usuario que haya iniciado sesión.

Integra de forma predeterminada protección contra amenazas comunes, como CSRF (Cross-Site Request Forgery).

\section{Thymeleaf}

Es un motor de plantillas para HTML, para permitir la creación de código HTML de forma dinámica.
Es un motor sencillo de usar, con expresiones básicas, que nos permite utilizar los objetos que enviamos para generar más o menos código, mostrar u ocultar opciones, \dots
Su curva de aprendizaje no es sencilla, pero dedicándole un poco de tiempo, se ahorra mucho a la hora de generar el código de las páginas web.
El uso de los \textit{fragments} es un gran aliado para ahorrarnos líneas de código.

\section{HTML}

Lenguaje utilizado en la creación de las páginas web, en particular HTML5, que tiene algunas funcionalidades básicas ya implementadas, como por ejemplo el uso de "required" en campos de formulario para que sea obligatorio rellenar los campos si necesidad de utilizar javascript para el control.

\section{MySQL}

Se utiliza MySQL como base de datos para almacenar los usuarios y roles.
Se ha decidido usar una base de datos porque su integración con Spring Security es muy sencilla y funcional. 
A través de los repositorios CRUD que nos proporciona Spring JPA podemos gestionar las tablas y registros de forma ágil.

\section{Docker}

\textbf{Docker} consiste en tener microservicios corriendo en una máquina, si necesidad de tener una gran infraestructura montada.

He utilizado docker tanto para el servidor de MySQL como para la propia aplicación Web, de esta forma con 2 simples docker, un docker-compose (fichero de configuración de dockers que permite descargarse, crear, parar o lanzar dockers) podemos tener toda la aplicación completa corriendo en cualquier entorno, ya que docker funciona en los principales sistemas operativos del mercado: Linux, Windows y macOS.

\subsection{Docker Compose}

Esta herramienta complementaria de docker, \textit{docker compose} \cite{web:dockercompose} permite definir y gestionar aplicaciones que requieran más de un contenedor, de forma conjunta y sencilla.
Se ha utilizado como base para la instalación del proyecto, explicando sus funcionalidades y permitiendo ciertas personalizaciones en el despligue de la aplicación.

\section{GitHub}

Se va utiliza GitHub para dar visibilidad al trabajo diario y constante en el proyecto, ya que permite realizar aportaciones incrementales de código, documentación, etc del TFG completo.

Es la herramienta por excelencia para llevar un control de versiones en cualquier proyecto, sea software o no.

\subsection{Repository}

Se ha creado un repositorio~\cite{github:repo} donde se subirá todo el TFG al completo, tanto la documentación de la memoria, como la aplicación de java.

\subsection{Releases}

En esta sección dentro del repositorio se han ido subiendo las diferentes versiones funcionales de la aplicación, para tener un lugar central desde el que descargarse la última versión ``liberada''.

\subsection{Packages}

En esta parte de GitHub se han subido las imágenes de docker funcionales de la aplicación, para pemitir un despliegue sencillo desde una ubicación centralizada.
Con un simple comando de \textit{docker pull url de la imagen} puede descargarte la imagen para empezar a utilizarla.

\subsection{Proyect}

Se ha creado un proyecto~\cite{github:proyect} relacionado, que sigue como la base la metodología Kanban, para crear nuevas tareas, relacionarlas con el repositorio y utilizar pizarras y paneles para ver el progreso, en lo que se está trabajando y lo ya completado.

\textbf{Se prescinde de esta herramienta al sustituirse por Zube.io}

\section{JSON}

Se van a utilizar ficheros con la estructura de un json para almacenar los cuestionarios completos, a los que se accederá desde la aplicación web de Java.

\section{Zube.io}

Se va a utilizar la versión gratuita de Zube.io, que permite generar Sprints, y gráficos de seguimiento, como \textit{Burndown} o \textit{Burnup}.

Gracias a la sencilla y completa integración con Github ofrece una versión más completa de la parte de Proyect de Github, por lo que he decidido utilizar esta plataforma para el seguimiento de las tareas en vez de la que ofrecía Github.

La versión gratuita permite integrar lo necesario del repositorio y del proyecto, por lo que no es necesario ampliar a versiones más completas.

Como pega, no permite hacerlo público para compartir su contenido, pero con los gráficos se suple esta limitación.

\section{GitPod}

Se ha estudiado la posibilidad de usar \textit{gitpod} para alojar una versión funcional de la aplicación, ya que diferentes alternativas como \textit{Heroku} o \textit{Render} ni permitían un fácil despliegue con docker-compose.yml ni daban la opción de una prueba real de forma gratuita.

A pesar de haber vinculado \textit{gitpod} con \textit{GitHub}, se limitaba a 50 horas el despliegue de una aplicación, pero eso no se incluye como parte del proyecto.








