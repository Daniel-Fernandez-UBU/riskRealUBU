\apendice{Documentación técnica de programación}

\section{Introducción}

En el anexo actual se va a describir toda la documentación técnica, así como la estructura de directorios que se ha seguido para almacenar el proyecto completo.
Una de las partes más importantes es la sección de \textit{Manual del Programador} ya que en ella se describen los pasos necesarios para instalar el entorno de desarrollo. 

\section{Estructura de directorios}

El repositorio del proyecto está formado por los siguientes directorios:

\begin{itemize}
	\item \textbf{/}: Contiene el archivo README, la copia de la licencia y el comprimido riskteal.zip, con la versión corta para la instalación de la aplicación.
	\item \textbf{/docs}: Contiene la memoria y anexos en formato .pdf y .tex y la bibliografía en formato .bib.
	\item \textbf{/docs/tex}: Contiene lo diferentes archivos.tex que se han utilizado en la generación de los anexos y la memoria.
	\item \textbf{/docs/img}: Contiene las imágenes utilizadas en la construcción de los anexos y la memoria.
	\item \textbf{/docs/javadoc}: Contiene la documentación de la aplicación en formato javadoc.
	\item \textbf{/riskreal}: Contiene el docker-compose.yml que se utiliza para la instalación de la aplicación así como las carpetas \textbf{dbdata} y \textbf{appFiles} necesarias para el proceso.
	\item \textbf{/java}: Contiene las carpetas necesarias para el funcionamiento del entorno de desarrollo de la aplicación, así como todo el código completo del proyecto.
	\item \textbf{/java/csvFiles}: Carpeta donde se guarda el score.csv con los resultados de los cuestionarios.
	\item \textbf{/java/images}: Carpeta donde se guardan las imágenes de los cuestionarios.
	\item \textbf{/java/jsonFiles}: Carpeta que contiene los cuestionarios pendientes de generar.
	\item \textbf{/java/jsonFiles/lang}: Carpeta que contiene los cuestionarios ya generados en la aplicación.
	\item \textbf{/java/mysql}: Carpeta que contiene la carpeta \textbf{dbdata}, que se utiliza como ubicación de persistencia de la base de datos del entorno de desarollo.
	\item \textbf{/java/riskrealApp}: Carpeta que contiene el proyecto de maven y Spring Boot completo de la aplicación.
	\item \textbf{/java/riskrealApp/src/main}: Contiene todas las clases principales de java necesarias para que funcione la aplicación.
	\item \textbf{/java/riskrealApp/src/resources}: Contiene los archivos de configuración de la aplicación y el fichero quiz\_schema.json, con la estructura que deben seguir todos los json que se vayan a cargar.
	\item \textbf{/java/riskrealApp/src/resources/i18n}: Contiene los archivos de internacionalización de la aplicación.
	\item \textbf{/java/riskrealApp/src/resources/templates}:  Contiene las plantillas html necesarias para que funcione cada menú de la aplicación.
	\item \textbf{/java/riskrealApp/src/resources/static}: Contiene los ficheros de personalización para que se muestren las páginas web en un formato correcto. 
\end{itemize}

\section{Manual del programador}

El manual del programador es la guía \textit{paso a paso} de las diferentes herramientas, configuraciones y peculiaridades que se deben seguir para poder montar el entorno de desarrollo, trabajar con el código fuente o recompilar una nueva versión de la aplicación.

Las imágenes que se pueden ver en los siguientes pasos han sido tomadas desde un portátil con el sistema operativo Windows 11 y con las versiones aquí indicadas de las diferentes aplicaciones, por lo tanto, en distintas versiones el contenido o la ubicación de algunas opciones pueden estar en diferentes apartados.

\subsection{Herramientas necesarias}

Para poder desplegar el entorno de desarrollo se necesita instalar las siguientes aplicaciones:

\begin{itemize}
	\item Java SE JDK 17.
	\item Docker.
	\item Git.
	\item Spring Tools 4 for Eclipse.
\end{itemize}

A continuación, se explica cómo instalar y configurar cada uno de ellos de forma correcta.

\subsection{Java SE JDK 17}

Es el lenguaje de programación utilizado para la realización de la aplicación web.
Se debe acceder al siguiente enlace \cite{web:JavaJDK}, seleccionar el sistema operativo y arquitectura correspondiente y seguir el asistente de instalación.

\subsection{Docker}

En el momento actual en el que se encuentra la informática y el interés por la optimización de recursos, Docker se ha posicionado como la plataforma principal de \textit{microservicios}.

Docker Desktop se puede descargar desde el enlace \cite{web:dockerDesktop}. Tiene versión para windows, linux y mac, por lo que hay que elegir el sistema operativo correcto y seguir los pasos del asistente de instalación.

\textbf{La instalación exige un reinicio del equipo}, que se recomienda hacer en este punto para poder seguir con el resto del manual de forma secuencial.

Una vez instalado, la aplicación sería similar a la siguiente:
\imagen{dockerDesktopMain}{Pantalla principal de Docker Desktop.}

Podemos cerrar la aplicación sin problema porque se queda en ejecución en segundo plano, pero para confirmarlo, podemos abrir \textit{el símbolo del sistema - cmd} o una consola de powershell y ejecutar el comando \textbf{docker version}, donde deberíamos obtener una imagen similar a la siguiente:
\imagen{dockerVersion}{Versión de docker en ejecución desde CMD de Windows.}

Lo importante de la imagen anterior es que aparezca la información de \textbf{Server: Docker Desktop}, si no apareciese, abrir la aplicación de Docker Desktop y comprobar que ahora sí tenemos la información del servidor.

\subsection{Git}

Git es la herramienta que nos permite recuperar todo el proyecto desde el repositorio web \cite{github:proyect}. 
Nos descargamos Git desde el siguiente enlace \cite{web:git}.
Seleccionamos la versión que se corresponda con nuestro sistema operativo y seguimos los pasos del asistente de instalación.

Una vez instalado, vamos a utilizar la herramienta \textbf{Git Bash}
\imagen{gitBash}{Aplicación Git Bash.}

Navegamos por los directorios hasta el directorio en el que queramos replicar el proyecto y a continuación, escribimos \textit{git clone https://github.com/Daniel-Fernandez-UBU/riskRealUBU}
\imagen{gitClone}{Comando Git Clone.}

\subsection{Spring Tools 4 for Eclipse}

\textit{Spring Tools 4 for Eclipse} es la herramienta desde la que se ha construido el proyecto de Spring al completo. 

Se puede obtener desde el siguiente enlace \cite{web:springtoolssuite}
\imagen{springTools4forEclipse}{Web de descarga de Spring Tools 4.}

En el caso de Windows, el fichero descargado es un \textit{.jar}. 

\begin{enumerate}
	\item Hacemos doble click sobre el fichero .jar descargado, para que se inicie la extracción automática del contenido.
	\begin{itemize}
		\item Si lo anterior no funciona, se tiene que realizar \textit{java -jar ``nombre-archivo-jar.jar''} para que se inicie.
	\end{itemize}
	\item En la carpeta en la que esté el fichero .jar se generará una nueva carpeta con nombre \textbf{sts-4.22.1.RELEASE}, o similar, dependiendo de la versión de la aplicación en el momento de la descarga.
	\item Movemos esa carpeta a la ubicación que nos interese, pues el programa de Spring Tools es \textit{portable}.
	\item Accedemos a la carpeta y ejecutamos \textbf{SpringToolSuite4.exe}
\end{enumerate}

Tras unos instantes en los que aparece la siguiente ventana:
\imagen{springToolsStart}{Spring Tools Suite 4 inicio.}

Nos pedirá que indiquemos la ubicación de nuestro espacio de trabajo, donde se guardarán por defecto todos los proyectos que creemos desde la aplicación.
\imagen{springToolsWork}{Spring Tools Suite 4 espacio de trabajo.}

Tras indicar nuestra ubicación preferida, pulsamos en \textit{Launch}.

Tras unos instantes en los que carga la aplicación, se nos abre la aplicación:

\imagen{springToolsIDE}{Spring Tools Suite 4 - IDE.}

Antes de importar el proyecto, vamos a instalar algunos plugins que pueden resultar de utilidad.

El proceso para la instalación de los plugins es el mismo, solo cambia el plugin a buscar.

Pasos a seguir para la instalación de cualquier plugin:
\begin{enumerate}
	\item Pulsamos en \textit{Help} --> \textit{Eclipse Marketplace\dots}
	\item Introducimos el nombre del plugin que queremos buscar.
	\item Pulsamos en \textit{Go} o le damos al \textit{Intro} en el teclado.
	\item Pulsamos en \textit{Install}.
	\item Seguimos los pasos, aceptando las diferentes ventanas de confirmación que nos aparecen.
	\item Tras terminarse la instalación del plugin, nos pide reiniciar Spring Tools Suite 4. \textbf{Consejo: Instalar todos los plugins y al final reiniciar la aplicación}
\end{enumerate}

\subsubsection{Thymeleaf}

Siguiendo los pasos anteriores, obtendríamos la siguiente ventana, donde ya solo quedaría darle a \textit{Install} y seguir los pasos del asistente de instalación.
\imagen{pluginThymeleaf}{Instalación del plugin de Thymeleaf en Spring Tools Suite.}

\subsubsection{Eclipse Web Developer}

Instalamos la versión \textit{Eclipse Enterprise Java and Web Developer Tools 3.33} que incluye los editores de html, json y css entre otros.
\imagen{pluginEnterpriseJava}{Instalación del plugin de Eclipse Enterprise Java and Web en Spring Tools Suite.}

Seleccionamos las características que vamos a instalar:
\imagen{pluginEnterpriseJavaFeatures}{Selección de características de Eclipse Enterprise Java and Web en Spring Tools Suite.}


\subsubsection{Eclipse Docker Tooling}

Este plugin viene instalado por defecto en la última versión disponible de Spring Tools 4 for Eclipse, pero si no apareciese en la sección de \textit{Installed} en \textit{Eclipse Marketplace}, también sería necesario instalarlo.

\imagen{pluginDocker}{Plugin de Docker en Spring Tools Suite.}


\subsubsection{Importación de nuestro proyecto}

Una vez hayamos finalizado con la instalación de los plugins anteriores, procederemos a importar nuestro proyecto.

En primer lugar seleccionamos \textbf{File --> Import\dots}

\imagen{fileImport}{Importar proyecto en Spring Tool Suite.}

En la nueva ventana que se muestra, navegamos hasta \textbf{Maven --> Existing Maven Projects --> Next}.

\imagen{importMaven}{Importación de proyecto Maven en Spring Tool Suite.}

A continuación, nos pedirá que busquemos la ruta donde tenemos el proyecto que queremos importar. \textbf{Tenemos que ir a la ruta sobre la que hayamos ejecutado el \textit{git clone} del paso previo}.

Accedemos a la ruta donde hemos hecho el clon del repositorio de Git, y dentro de la carpeta del proyecto, \textit{riskRealUBU}, accedemos a \textit{java --> riskRealApp}.

Si hemos seleccionado la ruta adecuada, deberá quedar de esta forma la importación, habiendo reconocido el fichero \textbf{pom.xml} de nuestro proyecto y a la espera de que pulsemos en \textit{Finish}.

\imagen{browseProject}{Ruta del proyecto indicada en Spring Tool Suite.}

Tras unos instantes, ya nos aparecería el proyecto correctamente cargado en nuestra aplicación.

\imagen{proyectImported}{Proyecto importado en Spring Tools Suite.}

\subsubsection{Iniciando la aplicación}

Tras haber completado correctamente todos los pasos anteriores, y teniendo \textit{Docker Server} funcionando, desde \textit{Spring Tools Suite} podemos ejecutar la aplicación.

Los puertos por defecto que usa son: 3306 para \textit{mysql} y 8088 para la aplicación web.
Sería recomendable no tener ninguno de los 2 en uso para el correcto despliegue de la aplicación sin necesidad de tener que cambiar opciones de configuración en \textit{application.properties}.

En la parte inferior izquierda de la interfaz de Spring Tool, tendremos \textit{Boot Dashboard} y si desplegamos \textit{local} nos aparecerá nuestra aplicación \textbf{riskrealApp}

Seleccionamos nuestra aplicación y pulsamos en el icono que tenemos justo encima con un \textit{cuadrado rojo y triángulo verde} para iniciarla.

\imagen{startApp}{Iniciar la aplicación en Spring Tools Suite.}

Tras iniciar la aplicación, si es la primera vez tardará un poco más porque tiene que descargarse de docker la imagen de MySQL, debería quedarse en un estado similar a este:

\imagen{appStarted}{Aplicación iniciada en Spring Tools Suite.}


Ya tenemos nuestro entorno de desarrollo configurado, con nuestra aplicación iniciada y accesible para poder probarla desde \url{http://localhost:8088}.

\section{Compilación, Generación de una nueva imagen, y Docker Compose}

\subsection{Compilación}

Para compilar el proyecto y generar una nueva imágen de docker de nuestra aplicación, tenemos que realizar los siguientes pasos:

\begin{enumerate}
	\item Ejecutar desde Spring Tools Suite 4: Botón derecho del ratón en el nombre del proyecto --> Run as\dots --> Maven build\dots --> Escribimos \textbf{spring-boot:build-image}
	\item Una vez termine ya tenemos la imagen cargada en docker --> Se puede comprobar con \textbf{docker images} desde el símbolo del sistema o en la aplicación de Docker Desktop.
\end{enumerate}

\subsection{Generación de una nueva imagen}

Tras lo anterior, podemos subir la imagen a nuestro repositorio de github de forma muy sencilla siguiendo estos pasos:

\begin{enumerate}
	\item Renombramos la imagen de docker siguiendo el patrón: \textbf{ghcr.io/<username de github>/<repositorio>/<nombre de imagen>:<version>} (la versión es opcional).
 	\item Por ejemplo: \textbf{docker image tag riskrealapp:latest ghcr.io/daniel-fernandez-ubu/riskrealubu/riskrealapp}
 	\item Crear un \textbf{Access Token (classic)} en nuestro perfil de GitHub --> Settings --> Developer Settings (Está abajo del todo)
 	\item Iniciar sesión en la terminal (o línea de comandos) desde donde vayamos a hacer el push con el comando:
 	\item \textbf{echo <token> | docker login ghcr.io -u USERNAME --password-stdin}
 	\item Hacer pull de nuestra imagen al repositorio:
 	\item \textbf{docker push ghcr.io/daniel-fernandez-ubu/riskrealubu/riskrealapp}. Al no poner la versión, utiliza \textit{latest} por defecto.
	\item Si hemos vinculado el repositorio de \textbf{package} con nuestro repositorio, ya al acceder vemos que hay una nueva imagen.
\end{enumerate}

\subsection{Fichero docker-compose.yml}\label{subsec:dockerComposeFile}

Es el fichero más importante de cara a lanzar una nueva versión de la aplicación, por ese motivo se ha decidido hacer una pequeña introduccion del contenido del fichero, para entender el motivo de cada línea y para que cada programador pueda configurarlo a su gusto.

El contenido del fichero es el siguiente:
\imagen{dockerComposeYML}{Contenido del fichero docker-compose.yml}

A continuación, explicamos cada una de sus líneas:
\begin{itemize}
	\item services: --> Etiqueta que se utiliza para definir los servicios o dockers que se van a crear.
	\begin{itemize}
		\item db: --> Servicio para el docker que contiene la base de datos.
		\begin{itemize}
			\item image: mysql --> Imagen de la base de datos que se va a utilizar.
			\item container\_name: riskreal-db-1 --> Nombre del docker que se va a crear.
			\item restart: always --> Reinicia el docker si se genera un error.
			\item environment: --> Definimos varias variables de entorno de la imagen que vamos a lanzar.
			\begin{itemize}
				\item MYSQL\_DATABASE: appRiskRealUbu --> Nombre de la base de datos que se va a crear.
				\item MYSQL\_ROOT\_PASSWORD: rootroot --> Contraseña de root para acceder a la base de datos.
			\end{itemize}
			\item expose: --> Puertos internos que se van a abrir, para que el resto de contenedores de la misma red puedan acceder.
			\begin{itemize}
				\item - '3306' --> Se habilita el puerto 3306 para acceder a este docker por la red interna.
			\end{itemize}
			\item ports: --> Puertos que se publican del docker para acceder a él desde fuera de la red interna, es decir, desde el equipo anfitrión.
			\begin{itemize}
				\item '30306:3306' --> Permite el acceso a la base de datos desde el puerto 30306.
			\end{itemize}
			\item volumes: --> Se definen los directorios que queremos que sean persistentes, es decir, que tras el reinicio o eliminación del docker, se mantengan.
			\begin{itemize}
				\item ./dbdata:/var/lib/mysql --> Se utiliza la carpeta local dbdata para almacenar la base de datos completa. Si dbdata tuviese una base de datos MySQL, el docker la utilizaría al iniciar.
			\end{itemize}
			\item networks: --> Define la red interna del docker.
			\begin{itemize}
				\item - riskreal-network --> Nombre de la red interna.
			\end{itemize}
			\item healtcheck: --> Comprobación adicional para saber cuando ha terminado de levantar la base de datos en el docker.
			\begin{itemize}
				\item test: --> Se lanza un ping con mysqladmin a localhost para saber cuando ha iniciado la base de datos.
				\item interval: 30s --> Define el intervalo de tiempo en el que se comprueba el estado.
				\item timeout: 10s --> Indica el intervalo de tiempo que espera antes de indicar que no hay respuesta del servidor.
				\item retries: 5 --> Número de veces que lo reintenta antes de indicar que el docker no ha iniciado correctamente.
			\end{itemize}
		\end{itemize}
		\item webapp: --> Servicio para el docker que contiene la aplicación web.
		\begin{itemize}
			\item image: ghcr.io/daniel-fernandez-ubu/riskrealubu/riskrealapp:v1 --> Imagen de docker que se usará. Imagen propia de la Spring Boot App.
			\item container\_name: riskreal-webapp-1 --> Nombre del docker que se va a crear.
			\item environment: --> Definimos varias variables de entorno de la imagen que vamos a lanzar.
				\begin{itemize}
				 	\item JSON\_QUIZ\_FILE\_PATH: /opt/jsonFiles --> Ruta en la que se guardan los json con los cuestionarios que se quieren cargar en la aplicación.
					\item JSON\_QUIZ\_FILE\_PATH\_LANG: /opt/jsonFiles/lang --> Ruta en la que se guardarán los cuestionarios ya cargados en la aplicación.
					\item CSV\_SCORE\_PATH: /opt/csvFiles/scores.csv --> 
- Ruta y nombre del fichero de resultados que se rellenará conforme se vayan realizando cuestionarios.
					 \item SPRING\_DATASOURCE\_URL: \\ jdbc:mysql://db:3306/appRiskRealUbu --> Dirección de la base de datos. 
					 \item SPRING\_DATASOURCE\_PASSWORD: rootroot --> Contraseña de root para acceder a la base de datos.
					 \item SPRING\_DATASOURCE\_DRIVER\_CLASS\_NAME: \\com.mysql.cj.jdbc.Driver --> Driver de conexión a la base de datos.
					 \item SERVER\_PORT: 8088 --> Puerto en el que se publicará la aplicación.
				 	 \item ADMIN\_USERNAME: admin@riskreal.com --> Usuario administrador por defecto.
				 	 \item ADMIN\_PASSWORD: admin --> Contraseña por defecto del usuario administrador.
  				 	 \item TEST\_QUIZ: test\_quiz\_ --> Nombre por defecto de los archivo json de test para usuarios anónimos.
				\end{itemize}
			\item ports: --> Puertos que se publican del docker para acceder a él desde fuera de la red interna, es decir, desde el equipo anfitrión.
			\begin{itemize}
				\item '8088:8088' --> Redirecciona el puerto 8088 del servidor anfitrión al puerto 8088 interno del docker, en el que se ha publicado la aplicación.
			\end{itemize}
			\item volumes: --> Se definen los directorios que queremos que sean persistentes, es decir, que tras el reinicio o eliminación del docker, se mantengan.
			\begin{itemize}
				\item ./appFiles:/opt --> Se utiliza para indicar la ruta de los cuestionarios que queremos cargar, los ya almacenados y el fichero de resultados.
				\item ./appFiles/images:/workspace/BOOT-INF/classes/static/images --> \\Se utiliza para indicar la ruta donde tenemos las imágenes de los cuestionarios.
			\end{itemize}
			\item networks: --> Define la red interna del docker.
			\begin{itemize}
				\item - riskreal-network --> Nombre de la red interna.
			\end{itemize}
			\item depends\_on: --> Se definen los servicios de los que depende el inicio de este docker.
			\begin{itemize}
				\item db: --> Servicio del que depende.
				\item condition: service\_healthy --> Necesita que se haya iniciado de forma correcta la base de datos para iniciar el docker.
			\end{itemize}
		\end{itemize}
	\end{itemize}
	\item neworks: --> Sección en la que se definen las redes.
	\begin{itemize}
		\item riskreal-network: --> Nombre de la red definida.
	\end{itemize}
\end{itemize}

\section{Pruebas del sistema}

Dada la dificultad de realizar pruebas automáticas en una aplicación web, se ha trabajado con pruebas manuales.

La mayoría de las funcionalidades se basan en el registro o modificación de datos de usuarios, donde los propios campos de los ficheros \textit{HTML} tienen las restricciones necesarias para que se cumpla con lo esperado.

Las funcionalizades sobre las que se han realizado pruebas son:

\begin{itemize}
	\item La carga de un archivo json desde el sistema de archivos a la aplicación.
	\begin{itemize}
		\item Si no se selcciona un archivo json la aplicación te avisa y no deja cargarlo.
		\item Si se selecciona un archivo json pero su estructura es incorrecta, no lo carga y avisa.
	\end{itemize}
	\item El envío de un correo electrónico de recuperación de contraseñas.
	\begin{itemize}
		\item Si el correo electrónico introducido está en la base de datos, se envía el email.
		\item Si no lo está, se informa indicando que no es correcto.
		\item Si ha sucedido algún error que no permita enviar el correo, se informa indicando de esta situación: Solo podría darse si hubiese problemas con el servidor de correo o la cuenta utilizada para el envío.
	\end{itemize}
\end{itemize}
