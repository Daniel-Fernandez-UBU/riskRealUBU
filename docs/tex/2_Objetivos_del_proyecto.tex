\capitulo{2}{Objetivos del proyecto}

A continuación, se detalla los diferentes objetivos que han impulsado la realización del proyecto.

\section{Objetivos generales}

\begin{itemize}
	\item Desarrollar una aplicación web que permita la realización de cualquier tipo de cuestionario sobre habilidades blandas.
	\item Crear una aplicación sencilla y portable, fácilmente utilizable, y que permita cierta configuración adicional de base.
	\item Almacenar los resultados en un archivo estructurado y sencillo de analizar.
	\item Permitir que la aplicación pueda llegar a la mayoría de los países del entorno, estando disponible en varios idiomas.
\end{itemize}



\section{Objetivos técnicos}


\begin{itemize}
	\item Utilizar un \textit{framework} de java para el desarrollo de la aplicación.
	\item Convertir la aplicación web en lo más portable posible, utilizando tecnologías como \textit{docker}.
	\item Aplicar la arquitectura \textit{modelo-vista-controlador}. separando la parte del código que se encarga de presentar la imagen de la que se encarga de obtener los datos.
	\item Incluir la conexión a una base de datos para almacenar los datos de los usuarios de forma persistente.
	\item Utilizar GitHub como plataforma de control de versiones.
	\item Utilizar \textit{maven} como herramienta de automatización en la construcción de la aplicación y la imagen de \textit{docker}.
\end{itemize}

\section{Objetivos personales}

\begin{itemize}
	\item Adentrarme en el \textit{framework de Spring Boot}. algo totalmente desconocido para mí hasta el momento, y permitirme ver lo sencillo que es la creación de una aplicación web, una vez que se conocen todas sus bondades.
	\item Explorar el ecosistema completo de la programación web, tanto por lo anterior como por el uso de motores de plantillas y de generación de código \textit{HTML} de forma dinámica con \textit{Thymeleaf}.
	\item Profundizar en \textit{docker} y su ecosistema, una tecnología con gran auge en el mercado laboral actual.
	\item La creación de una aplicación que tenga claramente diferenciado el \textit{front-end} (páginas \textit{HTML}) del \textit{back-end} (código en java) y con persistencia fuera de la aplicación (base de datos y almacenamiento local o centralizado en el equipo anfitrión).
\end{itemize}