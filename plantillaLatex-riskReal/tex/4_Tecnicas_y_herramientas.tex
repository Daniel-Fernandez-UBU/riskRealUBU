\capitulo{4}{Técnicas y herramientas}

Esta parte de la memoria tiene como objetivo presentar las técnicas metodológicas y las herramientas de desarrollo que se han utilizado para llevar a cabo el proyecto. Si se han estudiado diferentes alternativas de metodologías, herramientas, bibliotecas se puede hacer un resumen de los aspectos más destacados de cada alternativa, incluyendo comparativas entre las distintas opciones y una justificación de las elecciones realizadas. 

\section{Eclipse}

Herramienta de desarrollo (IDE) que se utilizará para llevar a cambo el proyecto.
A pesar de haber otras similares, como InteliJ, he preferido utilizar Eclipse al estar más fmailiarizado con ella por ser la utilizada en otras asignaturas de la carrera para la programación en Java.

\section{LaTeX}
El editor de texto para generar toda la documentación relacionada con el trabajo de fin de grado será LaTeX.
Aún estándo acostumbrado a utilizar Microsoft Word para este tipo de tareas, veo en utilizar LaTeX una oportunidad de aprender.
La curva de aprendizaje es más grande que con Word, pero el resultado merece la pena, porque permite utilizar los formatos de forma sencilla, un aspecto en el que Word está en clara desventaja.
\section{Spring Tools}

Para la aplicación web en java se utilizará el Framework de Spring.
Es mi primera vez con un framework en java, y para aprender conceptos de forma más rápida y con ejemplos he realizado un curso en Udemy \cite{udemy:eliseo} para dar un acelerón en el aprendizaje.

\section{GitHub}

Se va a utilizar GitHub para dar visibilidad al trabajo diario y constante en el proyecto, ya que permite realizar aportaciones incrementales de código, documentación, etc del TFG completo.

\subsection{Repository}

Se ha creado un repositorio~\cite{github:repo} donde se subirá todo el TFG al completo, tanto la documentación de la memoria, como la aplicación de java.

\subsection{Proyect}

Se ha creado un proyecto~\cite{github:proyect} relacionado, que sigue como la base la metodología Kanban, para crear nuevas tareas, relacionarlas con el repositorio y utilizar pizarras y paneles para ver el progreso, en lo que se está trabajando y lo ya completado.

\textbf{Se prescinde de esta herramienta al sustituirse por Zube.io}

\section{JSON}

Se van a utilizar ficheros con la estructura de un json para almacenar los cuestionarios completos, a los que se accederá desde la aplicación web de Java.

\section{Zube.io}

Se va a utilizar la versión gratuita de Zube.io, que permite generar Sprints, y gráficos de seguimiento, como Burndowns o Burnups.

Gracias a la sencilla y completa integración con Github ofrece una versión más completa de la parte de Proyect de Github, por lo que he decidido utilizar esta plataforma para el seguimiento de las tareas en vez de la que ofrecía Github.

La versión gratuita permite integrar lo necesario del repositorio y del proyecto, por lo que no es necesario ampliar a versiones más completas.







