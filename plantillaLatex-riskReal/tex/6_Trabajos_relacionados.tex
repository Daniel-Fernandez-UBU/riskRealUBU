\capitulo{6}{Trabajos relacionados}

Este apartado sería parecido a un estado del arte de una tesis o tesina. En un trabajo final grado no parece obligada su presencia, aunque se puede dejar a juicio del tutor el incluir un pequeño resumen comentado de los trabajos y proyectos ya realizados en el campo del proyecto en curso. 

\section{Web RiskReal.eu}
Es la página web \cite{web:riskreal} que se ha utilizado como base para la creación del nuevo proyecto.
La aplicación se basa en la utilización de diferentes cuestionarios para evaluar "soft skills" de trabajadores.

\subsection{Descripción general}
Desde la página principal se ofrece una descripción básica de lo que se puede hacer.
Indica que con diversos escenarios de test, se pueden evaluar de forma eficiente las diferentes "soft skills".

Consta de dos páginas base, "Inicio", que consideraría la parte "abierta" de la web, y "Cursos", con acceso restringido solo para usuarios registrados.

\subsection{Parte abierta}
Permite realizar un cuestionario propio, para obtener una evaluación orientativa de en qué estado nos encontramos en cuanto a habilidades blandas; y un test sobre un posible escenario, donde el "trabajador" va contestando en función de diferentes situaciones.

\subsection{Parte privada}
Aunque no tengo acceso a esta parte, es la que permite a las empresas generar sus propios cuestionarios para evaluar a sus trabajadores.
Permite esa personalización necesaria para que en función del sector o lo que se quiera evaluar, se pueda generar algo específico.



