\apendice{Documentación técnica de programación}

\section{Introducción}

En el anexo actual se va a describir toda la documentación técnica, así como la estructura de directorios que se ha seguido para almacenar el proyecto completo.
Una de las partes más importantes es la sección de \textit{Manual del Programador} ya que en ella se describen los pasos necesarios para instalar el entorno de desarrollo. 

\section{Estructura de directorios}

El repositorio del proyecto está formado por los siguientes directorios:

\section{Manual del programador}

El manual del programador es la guía \textit{paso a paso} de las diferentes herramientas, configuraciones y peculiaridades que se deben seguir para poder montar el entorno de desarrollo, trabajar con el código fuente o recompilar una nueva versión de la aplicación.

Las imágenes que se pueden ver en los siguientes pasos han sido tomadas desde un portátil con el sistema operativo Windows 11 y con las versiones aquí indicadas de las diferentes aplicaciones, por lo tanto, en distintas versiones el contenido o la ubicación de algunas opciones pueden estar en diferentes apartados.

\subsection{Herramientas necesarias}

Para poder desplegar el entorno de desarrollo se necesita instalar las siguientes aplicaciones:

\begin{itemize}
	\item Java SE JDK 17.
	\item Docker.
	\item Git.
	\item Spring Tools 4 for Eclipse.
\end{itemize}

A contunación, se explica cómo instalar y configurar cada uno de ellos de forma correcta.

\subsection{Java SE JDK 17}

Es el lenguaje de programación utilizado para la realización de la aplicación web.
Se debe acceder al siguiente enlace \cite{web:JavaJDK}, seleccionar el sistema operativo y arquitectura correspondiente y seguir el asistente de instalación.

\subsection{Docker}

En el momento actual en el que se encuentra la informática y el interés por la optimización de recursos, Docker se ha posicionado como la plataforma principal de \textit{microservicios}.

Docker Desktop se puede descargar desde el enlace \cite{web:dockerDesktop}. Tiene versión para windows, linux y mac, por lo que hay que elegir el sistema operativo correcto y seguir los pasos del asistente de instalación.

\textbf{La instalación exige un reinicio del equipo}, que se recomienda hacer en este punto para poder seguir con el resto del manual de forma secuencial.

Una vez instalado, la aplicación sería similar a la siguiente:
\imagen{dockerDesktopMain}{Pantalla principal de Docker Desktop.}

Podemos cerrar la aplicación sin problema porque se queda en ejecución en segundo plano, pero para confirmarlo, podemos abrir \textit{el símbolo del sistema - cmd} o una consola de powershell y ejecutar el comando \textbf{docker version}, donde deberíamos obtener una imagen similar a la siguiente:
\imagen{dockerVersion}{Versión de docker en ejecución desde CMD de Windows.}

Lo importante de la imagen anterior es que aparezca la información de \textbf{Server: Docker Desktop}, si no apareciese, abrir la aplicación de Docker Desktop y comprobar que ahora si tenemos la información del servidor.

\subsection{Git}

Git es la herramienta que nos permite recuperar todo el proyecto desde el repositorio web \cite{github:proyect}. 
Nos descargamos Git desde el siguiente enlace \cite{web:git}.
Seleccionamos la versión que se corresponda con nuestro sistema operativo y seguimos los pasos del asistente de instalación.

Una vez instalado, vamos a utilizar la herramienta \textbf{Git Bash}
\imagen{gitBash}{Aplicación Git Bash.}

Navegamos por los directorios hasta el directorio en el que queramos replicar el proyecto y a continuación, escribimos \textit{git clone https://github.com/Daniel-Fernandez-UBU/riskRealUBU}
\imagen{gitClone}{Comando Git Clone.}

\subsection{Spring Tools 4 for Eclipse}

\textit{Spring Tools 4 for Eclipse} es la herramienta desde la que se ha construido el proyecto de Spring al completo. 

Se puede obtener desde el siguiente enlace \cite{web:springtoolssuite}
\imagen{springTools4forEclipse}{Web de descarga de Spring Tools 4.}

En el caso de Windows, el fichero descargado es un \textit{.jar}. 

\begin{enumerate}
	\item Hacemos doble click sobre el fichero .jar descargado, para que se inicie la extracción automática del contenido.
	\begin{itemize}
		\item Si lo anterior no funciona, se tiene que realizar \textit{java -jar ``nombre-archivo-jar.jar''} para que se inicie.
	\end{itemize}
	\item En la carpeta en la que esté el fichero .jar se generará una nueva carpeta con nombre \textbf{sts-4.22.1.RELEASE}, o similar, dependiendo de la versión de la aplicación en el momento de la descarga.
	\item Movemos esa carpeta a la ubicación que nos interese, pues el programa de Spring Tools es \textit{portable}.
	\item Accedemos a la carpeta y ejecutamos \textbf{SpringToolSuite4.exe}
\end{enumerate}

Tras unos instantes en los que aparece la siguiente ventana:
\imagen{springToolsStart}{Spring Tools Suite 4 inicio.}

Nos pedirá que indiquemos la ubicación de nuestro espacio de trabajo, donde se guardarán por defecto todos los proyectos que creemos desde la aplicación.
\imagen{springToolsWork}{Spring Tools Suite 4 espacio de trabajo.}

Tras indicar nuestra ubicación preferida, pulsamos en \textit{Launch}.

Tras unos instantes en los que carga la aplicación, se nos abre la aplicación:

\imagen{springToolsIDE}{Spring Tools Suite 4 - IDE.}

Antes de importar el proyecto, vamos a instalar algunos plugins que pueden resultar de utilidad.

El proceso para la instalación de los plugins es el mismo, solo cambia el plugin a buscar.

Pasos a seguir para la instalación de cualquier plugin:
\begin{enumerate}
	\item Pulsamos en \textit{Help} --> \textit{Eclipse Marketplace\dots}
	\item Introducimos el nombre del plugin que queremos buscar.
	\item Pulsamos en \textit{Go} o le damos al \textit{Intro} en el teclado.
	\item Pulsamos en \textit{Install}.
	\item Seguimos los pasos, aceptando las diferentes ventanas de confirmación que nos aparecen.
	\item Tras terminarse la instalación del plugin, nos pide reiniciar Spring Tools Suite 4. \textbf{Consejo: Instalar todos los plugins y al final reiniciar la aplicación}
\end{enumerate}

\subsubsection{Thymeleaf}

Siguiendo los pasos anteriores, obtendríamos la siguiente ventana, donde ya solo quedaría darle a \textit{Install} y seguir los pasos del asistente de instalación.
\imagen{pluginThymeleaf}{Instalación del plugin de Thymeleaf en Spring Tools Suite.}

\subsubsection{Eclipse Web Developer}

Instalamos la versión \textit{Eclipse Enterprise Java and Web Developer Tools 3.33} que incluye los editores de html, json y css entre otros.
\imagen{pluginEnterpriseJava}{Instalación del plugin de Eclipse Enterprise Java and Web en Spring Tools Suite.}

Seleccionamos las características que vamos a instalar:
\imagen{pluginEnterpriseJavaFeatures}{Selección de características de Eclipse Enterprise Java and Web en Spring Tools Suite.}


\subsubsection{Eclipse Docker Tooling}

Este plugin viene instalado por defecto en la última versión disponible de Spring Tools 4 for Eclipse, pero si no apareciese en la sección de \textit{Installed} en \textit{Eclipse Marketplace}, también sería necesario instalarlo.

\imagen{pluginDocker}{Plugin de Docker en Spring Tools Suite.}


\subsubsection{Importación de nuestro proyecto}

Una vez hayamos finalizado con la instalación de los plugins anteriores, procederemos a importar nuestro proyecto.

En primer lugar seleccionamos \textbf{File --> Import\dots}

\imagen{fileImport}{Importar proyecto en Spring Tool Suite.}

En la nueva ventana que se muestra, navegamos hasta \textbf{Maven --> Existing Maven Projects --> Next}.

\imagen{importMaven}{Importación de proyecto Maven en Spring Tool Suite.}

A continuación, nos pedirá que busquemos la ruta donde tenemos el proyecto que queremos importar. \textbf{Tenemos que ir a la ruta sobre la que hayamos ejecutado el \textit{git clone} del paso previo}.

Accedemos a la ruta donde hemos hecho el clon del repositorio de Git, y dentro de la carpeta del proyecto, \textit{riskRealUBU}, accedemos a \textit{java --> riskRealApp}.

Si hemos seleccionado la ruta adecuada, deberá quedar de esta forma la importación, habiendo reconocido el fichero \textbf{pom.xml} de nuestro proyecto y a la espera de que pulsemos en \textit{Finish}.

\imagen{browseProject}{Ruta del proyecto indicada en Spring Tool Suite.}

Tras unos instantes, ya nos aparecería el proyecto correctamente cargado en nuestra aplicación.

\imagen{proyectImported}{Proyecto importado en Spring Tools Suite.}

\subsubsection{Iniciando la aplicación}

Tras haber completado correctamente todos los pasos anteriores, y teniendo \textit{Docker Server} funcionando, desde Spring Tools Suite podemos ejecutar la aplicación.

\textbf{Los puertos por defecto que usa son: 3306 para mysql y 8088 para la aplicación web, sería recomendable no tener ninguno de los 2 en uso, para el correcto despliegue de la aplicación sin necesidad de tener que cambiar opciones de configuración en \textit{application.properties}.}

En la parte inferior izquierda de la interfaz de Spring Tool, tendremos \textit{Boot Dashboard} y si desplegamos \textit{local} nos aparecerá nuestra aplicación \textbf{riskrealApp}

Seleccionamos nuestra aplicación y pulsamos en el icono que tenemos justo encima con un \textit{cuadrado rojo y triángulo verde} para iniciarla.

\imagen{startApp}{Iniciar la aplicación en Spring Tools Suite.}

Tras iniciar la aplicación, si es la primera vez tardará un poco más porque tiene que descargarse de docker la imagen de MySQL, debería quedarse en un estado similar a este:

\imagen{appStarted}{Aplicación iniciada en Spring Tools Suite.}


Ya tenemos nuestro entorno de desarrollo configurado, con nuestra aplicación iniciada y accesible para poder probarla desde \url{http://localhost:8088}.



\section{Pruebas del sistema}
