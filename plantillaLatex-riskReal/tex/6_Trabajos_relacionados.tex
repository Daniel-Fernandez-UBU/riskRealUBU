\capitulo{6}{Trabajos relacionados}

Este apartado sería parecido a un estado del arte de una tesis o tesina. En un trabajo final grado no parece obligada su presencia, aunque se puede dejar a juicio del tutor el incluir un pequeño resumen comentado de los trabajos y proyectos ya realizados en el campo del proyecto en curso. 

\section{Web RiskReal.eu}
Es la página web \cite{web:riskreal} que se ha utilizado como base para la creación del nuevo proyecto.
La aplicación se basa en la utilización de diferentes cuestionarios para evaluar "soft skills" de trabajadores.

\subsection{Descripción general}
Desde la página principal se ofrece una descripción básica de lo que se puede hacer.
Indica que con diversos escenarios de test, se pueden evaluar de forma eficiente las diferentes "soft skills".

Consta de dos páginas base, \textit{Inicio}, que consideraría la parte \textit{abierta} de la web, y \textit{Cursos}, con acceso restringido solo para usuarios registrados.

\subsection{Parte abierta}
Permite realizar un cuestionario propio, para obtener una evaluación orientativa de en qué estado nos encontramos en cuanto a habilidades blandas; y un test sobre un posible escenario, donde el "trabajador" va contestando en función de diferentes situaciones.

\subsection{Parte privada}
Se realiza un intento de registro con los siguentes datos:
\imagen{dataRiskRealAppRegister}{Datos de registro de prueba.}{.5}
Tras ello, no se recibe ningún correo de confirmación ni se consigue acceder a la propia aplicación.
Debido al contratiempo anterior, no se puede analizar más en detalle como es la parte "privada" de la aplicación.
Tampoco se permite el registro en la zona de empresas, obteniendo el mensaje que se puede ver en la siguiente imagen:
\imagen{errorRiskRealAppRegister}{Error tras enviar el formulario de \textit{Ir a la zona para empresas}.}{.5}





