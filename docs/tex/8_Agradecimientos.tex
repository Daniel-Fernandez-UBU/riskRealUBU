\capitulo{8}{Agradecimientos}

Quería utilizar un capítulo a parte para expresar mis agradecimientos personales, tanto por este Trabajo de Fin de Grado, como por estos años en la Universidad de Burgos.

Mi primera experiencia con una universidad fue hace más de 12 años, donde empecé por primera vez con Ingeniería Informática de forma presencial, pero no era el momento ni el lugar para llevar a cabo esa tarea.

Tengo que decir, que aún siendo en modalidad \textit{online}, la cercanía que he sentido durante estos 4 años y medio que llevo en la universidad me ha hecho sentir por momentos que estaba en persona en las clases.

Las facilidades que he encontrado en esta universidad para poder quitarme la espinita de cursar Ingeniería Informática y llegar hasta 4º, tras mucho esfuerzo y tras mucho aprendizaje, me llenan de emoción y de orgullo de decir que he estudiado en la Universidad de Burgos.

El agradacimiento está dirigido a todo el personal de la universidad, ya que a pesar de la distancia, siempre he recibido un trato rápido y cercano.

Me llevo la experiencia y la suerte de haber tenido grandes docentes, dispuestos a responder a un correo electrónico un fin de semana, o fuera de su jornada laboral, de hacer tutorías cuando mejor nos venía a todos los alumnos, contando con que la mayoría trabajamos y acababan siendo a última hora de la tarde.

La profesionalidad y vocación del personal de la universidad de los más altos que he vivido, muchas gracias por todos estos años.

Por último, y no menos importante, quiero agradecer a Raúl Marticorena Sánches, mi tutor en este Trabajo de Fin de Grado, y mi profesor en algunas de las asignaturas del grado, su especial dedicación y compromiso con los alumnos.

Sin pretender extenderme mucho más, mi más sincero agradecimiento a la Universidad de Burgos.


