\capitulo{7}{Conclusiones y líneas de trabajo futuras}

En este capítulo se exponen las conclusiones derivadas del trabajo del proyecto, así como las líneas de trabajo futuras por las que se puede dar continuidad al proyecto.

\section{Conclusiones}

Tras el desarrollo del proyecto se obtienen las siguientes conclusiones:

\begin{itemize}
	\item Se ha cumplido con el objetivo general del proyecto, teniendo ahora disponible una aplicación web que permite la realización y gestión de cuestionarios.
	\item Haber utilizado tecnologías como Spring Boot y Docker, le dan al proyecto un aura de modernidad, al ser tecnologías que están en auge dentro del ámbito del desarrollo de aplicaciones.
	\item He podido aplicar conocimientos de gran parte de lo aprendido durante el grado, y aprender sobre muchos otros en los que no contaba con ninguna experiencia previa.
	\item Integrar la creación del proyecto con herramientas de control de versiones ha facilitado el desarrollo incremental del proyecto, así como un \textit{know-how} de buenas prácticas para cualquier proyecto \textit{software} que tenga que desarrollar en el futuro.
	\item Estimar el tiempo total dedicado a pruebas y prototipos de la aplicación es muy complicado, ya que en algunos momentos me he atascado con parte de la programación y he tenido que probar varias alternativas hasta dar con la solución. No obstante, pero el uso de la metodología ágil para la gestión de las tareas ha facilitado el seguimiento del progreso.
\end{itemize}

\section{Líneas de trabajo futuras}

La entrega del Trabajo de Fin de Grado solo es el inicio del proyecto, ya que su desarrollo continúa.

A continuación, se resume el camino a seguir del proyecto:

\begin{itemize}
	\item Integrar en la página web la posibilidad de editar los cuestionarios ya cargados, sin tener que actualizar el archivo \textit{JSON} y volver a cargarlo.
	\item Integrar la aplicación con una \textit{API} de traducción que permita generar de forma automática los cuestionarios en otros idiomas.
	\item Integrar la posibilidad de crear cuestionarios desde cero desde la propia página web, contando con la ventaja de tener un modelo de datos ya definido.
	\item Se puede considerar la opción de migrar la aplicación a plataformas móviles, para no depender de la conexión a internet para acceder.
	\item Integrar un módulo de estadísticas, donde poder visualizar de forma gráfica los resultados obtenidos en los diferentes cuestionarios.
\end{itemize}


