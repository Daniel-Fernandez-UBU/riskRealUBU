\apendice{Plan de Proyecto Software}

\section{Introducción}

La fase de planificación es parte fundamental y necesaria en todos los proyectos.

Esta fase sive para ir evaluando la viabilidad del proyecto a lo largo de su vida, pudiendo prevenir o adelantarse a posibles inconvenientes que surgan durante la fase de desarrollo.
Tener medidas de forma correcta las tareas que se van realizando y su implicación en el total del proyecto es fundamental para un desarrollo sostenible, tanto económica como temporalmente.

Se ha decidido dividir esta fase en dos etapas:

\begin{itemize}
	\item
	Planificación temporal.
	\item
	Estudio de viabilidad.
\end{itemize}

La planificación temporal tratará sobre la gestión y aprovechamiento del tiempo, así como la división del proyecto en pequeños hitos, siguiendo una metodología ágil.

El estudio de viabilidad se descompone a su vez en:

\begin{itemize}
	\item
	Viabilidad económica: Inversión necesaria para acometer el proyecto y posible obtención de beneficios.
	\item
	Viabilidad legal: Todos los programas software conyevan unas implicaciones legales.
\end{itemize}

\section{Planificación temporal}

Para llevar a cabo una correcta planificación temporal se gestionado el proyecto siguiendo la metodología ágil que plantea Scrum, (poner enlace y bibliografía a Scrum) de forma "reducida" ya que el equipo de desarrolladores solo ha tenido un integrante, y las labores de "Scrum Master" y "Product Owner" han recaído en el tutor del TFG, el Sr. Raúl Marticorena.

Para el seguimiento del desarrollo incremental del proyecto software se ha divido la planificación en una serie de Sprints (enlace y referencia).

\subsection{Spring 1 - Base del proyecto}
\textit{Del 27/02/2024 al 11/03/2024}

Se estima una dedicación total de 20 horas de trabajo, siendo necesarias 24 horas para completar todas las tareas.

Temas tratados durante el Spring:

\begin{itemize}
	\item
	Uso e integración de zube.io con GitHub.
	\item
	Registro en https://app.riskreal.eu/ para analizar la parte "privada" de la aplicación.
	\item
	Schema json:
	\begin{itemize}
		\item
		Posibilidad de imagen en cada respuesta
		\item
		Atributo idioma en el cuestionario
	\end{itemize}
	Empezar a valorar y tener en cuenta la posibilidad de cuestionarios "Multi-idioma".
	\item
	Investigar sobre Thumeleaf y Spring Tools.
	\item
	Cargar un json de prueba en la app.
	\item
	Representación en una web del json.
	\item
	Posible estructura de directorios de la aplicación.
	\item
	Abordar la generación dinámica del contenido respecto al json.
\end{itemize}

Tareas tratadas cerradas a lo largo del Spring:

"imagen"

Gráfico de Burndown del Spring 1:

"imagen"


\end{itemize}

\subsection{Spring 2}

\subsection{Spring 3}

\subsection{Spring 4}

\subsection{Spring 5}

\subsection{Spring 6}


\section{Estudio de viabilidad}

\subsection{Viabilidad económica}

\subsection{Viabilidad legal}


